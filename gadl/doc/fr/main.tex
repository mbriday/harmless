\documentclass[11pt]{book}
\usepackage{geometry} % see geometry.pdf on how to lay out the page. There's lots.
\usepackage[frenchb]{babel}	%francais
\usepackage[utf8]{inputenc}	%francais
\geometry{a4paper} 
\usepackage{url}
\usepackage{graphicx}
\usepackage{listings}
\usepackage{tabularx}
\usepackage{hyperref}
% \geometry{landscape} % rotated page geometry

\newcommand{\cad}{c'est-à-dire~}
\newcommand{\bs}{\textbackslash}
\newcommand{\harmless}{Harmless}
\newcommand{\h}{Harmless}
\newcommand{\gadl}{Gadl}
\newcommand{\blocdo}{bloc {\em do \ldots\ end do}}
\newcommand{\blocsdo}{blocs {\em do \ldots\ end do}}

%command line options from Hevea to imagen
%for the generation of images with hevea (html output):
%-png       :generate png files (instead of gif)
%-pdf       : use pdflatex instead of latex (pdf files)
%-todir img : store image in img/
%HEVEA\@addimagenopt{-png -pdf -todir img} 
%say to hevea the image directory.
\providecommand{\heveaimagedir}{img}
\renewcommand{\heveaimagedir}{img}
%custom CSS file for HTML rendering.
%HEVEA\loadcssfile{../common/style.css}

% See the ``Article customise'' template for come common customisations

\title{Harmless \\Langage de description d'architecture matérielle \\ Documentation technique}
\author{M. Briday, R. Kassem, J.L. Béchennec}
%\date{} % delete this line to display the current date


%TODO: Rajouter les mots clés aussi dans le chapitre elements de base du langage situé
% \ref{keywords}
\lstdefinelanguage{Harmless} 
{morekeywords={model,include,port,device,architecture,write,shared,behavior,format,select,error,warning,component,void,every,memory,width,address,type,RAM,ROM,register,stride,read,program,counter,pipeline,stage,init,run,as,machine,BTB,FIFO,bypass,release,in,maps,to,stall,default,instruction,fetch,debug,big,little,endian,except,do,out,when,field,nop,slice,case,is,others,signed,or,syntax,switch,number,octal,decimal,hexadecimal,binary,suffix,prefix,timing,decode,size,jumpTaken,add,cycle,use,return,print,if,then,elseif,else,loop,while,end,true,false,ror,rol,cat}, 
sensitive=true, 
morecomment=[l]{--}, 
morestring=[b]"
}

\lstset{
  basicstyle=\small,
  stringstyle=\ttfamily, % typewriter type for strings
  showstringspaces=true %space in strings
  commentstyle=\ttfamily,
 % commentstyle=\itshape\color{vert},
  identifierstyle=\ttfamily\bfseries,
  keywordstyle=\ttfamily\bfseries\underbar,
  numbers=left, numberstyle=\tiny, stepnumber=1, numbersep=5pt, %line numbers
  breaklines=true,
  frame=lines, %bottom and top lines
  language=Harmless,
  defaultdialect=Harmless,
}

%%% BEGIN DOCUMENT
\begin{document}
\maketitle
\tableofcontents

%!TEX root = ./main.tex
%!TEX encoding = UTF-8 Unicode

\chapter{Installing \harmless, generation of a simulator}
\section{Development chain}
The development chain around the language of architecture description Harmless is composed of a number of tools that interact according to the schema\ref{fig:devTool}.
\begin{figure}		%% Small Example
  \begin{center}
    \includegraphics[width=0.8 \linewidth]{../common/images/devTools.pdf}
    \caption{Development chain}
    \label{fig:devTool}
  \end{center}
\end{figure}

These tools are:
\begin{itemize}
\item \gadl\ reads the file description in \harmless. This is the only tool necessary to generate a instruction set simulator (ISS);
\item  \texttt{p2a} and \texttt{a2cpp} are tools that are used for modeling the pipeline, when describing the internal architecture.  A description file of the pipeline is generated by \gadl\ according to the \h\ description,this is the input file of p2a. When the automaton has been generated by p2a, a second tool a2cpp translates the automaton description into $C++$ code, modeling of the pipeline.
\end{itemize}

This documentation focuses on the \gadl\ tool as the integration with the 2 other tools is in early stages of development.

\section{Building \gadl\ from sources}
The build process is made using a python script at the root of the archive : \texttt{buildHarmless.py}. Just configure your Internet access (proxy) if required and run the script. The build script:
\begin{itemize}
\item get the \texttt{galgas} compiler, as \gadl\ is written using the \texttt{galgas} language;
\item build the \texttt{p2a} and \texttt{a2cpp} tools;
\item build the library \texttt{libelf} that will be used to read .elf binary files for the simulators\footnote{\gadl\ is not compatible with the \texttt{libelf} library typically found on Linux distributions (including Ubuntu). Indeed, this library is linked to the simulator generated by \gadl, and we cannot impose a license to the generated code (That is the case with the GNU GPL. We use instead a library under the GNU LGPL. The library can be found on \url{http://www.mr511.de/software/english.html}}.
\end{itemize}

The script does not depends on specific tools (the only specific tools \texttt{galgas} is retrieved from Internet by the script). On Ubuntu/Debian systems, you should only install before basic development tools:
\texttt{apt-get install build-essential}


This script generates one big self-contained binary file that can generate a simulator from a description. The script is tested on Mac OS X, Linux x86/32 bits, Linux x86/64 bits and Linux ARMv6/32 bits. This has not been tested on Windows. 

Please report any problem during install to \url{mailto:harmless@irccyn.ec-nantes.fr}.

\subsection{Description editor}
A syntax file is provided for the \emph{Vim}\footnote{\url{http://www.vim.org}} editor, allowing syntax coloration. You only have to copy the syntax file to \texttt{~/.vim/syntax}. The directory have to be created if it does not yet exist.

With Mac OS X (10.5 required) an XCode project is provided. It can compile directly \gadl, but also build an application (Mac OS X bundle) to edit \harmless\ descriptions. It embeds a text editor with syntax highlighting and \gadl\ compiler.

\section{Dependencies for the simulator generation}

The generated simulator is written in $C++$  and can be used directly, without any dependency. However, a number of dependencies can be added to obtain additional services:
\begin{description}
\item[python] This allows to use the simulator in a Python script, and therefore does not require to recompile the simulator for each new scenario. The open source \texttt{swig}\footnote{\url{www.swig.org}} tool is used to generate the Python wrapper.
\item[libelf] This library allows to read \emph{elf} files, in addition to the default formats (SRecord Motorola and Intel Hex)
\end{description}

This section provides information on how to install these dependencies.

\subsection{Using the simulator with Python}
The simulator is generated in $C++$  and can be used in a Python script (version 2.5 and 2.6 tested). To enable the compilation of the Python wrapper, it is necessary to install the \texttt{Swig} tool:
\begin{itemize}
\item Linux Ubuntu: packages \texttt{swig} and \texttt{python2.5-dev}. The Python interpreter is installed by default.
\item Mac OS X: all is already included in the development tools (10.5).
\item MS Windows: Not tested.
\end{itemize}

\subsection{Library to read ELF files}
\label{sec:elf}
By default, \gadl\ can read the following file formats:
\begin{itemize}
\item Motorola SRecord (SREC)\footnote{\url{http://en.wikipedia.org/wiki/SREC_(file_format)}};
\item Intel Hex\footnote{\url{http://en.wikipedia.org/wiki/Intel_HEX}};
\end{itemize}
To add support for ELF format files \footnote{\url{http://en.wikipedia.org/wiki/Executable_and_Linkable_Format}}, an extra library is required. This last format has the advantage of encapsulating symbols (global variables, functions \ldots) in addition to executable code. 


\emph{Note}: \gadl\ is not compatible with the \texttt{libelf} library typically found on Linux distributions (including Ubuntu). Indeed, the library is linked to the simulator generated by \gadl, and it is not reasonable to impose the generated code to be under a license compatible with the GNU GPL\footnote{\url{http://fr.wikipedia.org/wiki/Licence\_publique\_generale\_GNU}}. We use a library under the GNU LGPL\footnote{\url{http://fr.wikipedia.org/wiki/Licence\_publique\_generale\_limitee\_GNU}}. The library can be found on \url{http://www.mr511.de/software/english.html}. To install it:
\begin{verbatim}
$ ./configure
$ make
$ sudo make install
\end{verbatim}

A local use is preferable on Linux distributions, to prevent conflicts with the other presents.

\emph{Note}: By default, the simulator is generated in 64-bit mode (compiler flag \texttt{-m64}), which gives much better simulation performance. In this case, the library libelf must also support 64-bit:
\begin{verbatim}
$ ./configure
$ make CFLAGS=-m64 LDFLAGS=-m64
$ sudo make install
\end{verbatim}

To stay in 32 bits, you should edit the generated \texttt{Makefile} and remove the \texttt{-m64} flag in \texttt{CFLAGS} and \texttt{LDFLAGS} (or better, update the Makefile.tpl in the template directory).



\section{Generation of a simulator}
The generation of the simulator is from the master file in the case of a description distributed over several files(voir section \ref{sec:plusieursFichiers}).

\gadl\ is a command line tool. Options availables are:
\begin{verbatim}
$ gadl --help
Compiled in 32 bits mode.
Usage : gadl [--help] [--version] [--lexical-analysis-only] [--parse-only] [-v] [--verbose] [--log-file-read] [--no-file-generation] [--Werror] [--detailled-error-messages] [-f] [--message-if-no-format] [-b] [--no-behavior] [-c] [--no-check] [-a] [--no-deasm] [-t] [--no-timing] [-s] [--speed] [-w] [--warn-trunc] [--max-errorss=number] [--max-warningss=number] [--model=string] [--templates=string] file...
Options:
 --help               Display help information
 --version            Display version
 --lexical-analysis-only
                      Perform only lexical analysis on input files
 --parse-only         Parse only input files
 -v,--verbose         Verbose Output
 --log-file-read      Log every file read
 --no-file-generation Do not generate any file
 --Werror             Treat warnings as errors
 --detailled-error-messages
                      Print detailled error messages
 -f,--message-if-no-format
                      get a message when an instruction signature found in a behavior or a syntax part have no corresponding format.
 -b,--no-behavior     does not take into account instruction behavior part. It is not possible to generate an instruction set simulator (ISS)
 -c,--no-check        Remove time consuming checks: orthogonality of instruction set.
 -a,--no-deasm        does not take into account instruction syntax part. It is not possible to generate a dissassembler
 -t,--no-timing       does not take into account instruction timing part. No cycle count. Simulation should be faster.
 -s,--speed           speed option. May be more difficult to debug. Inline component methods.
 -w,--warn-trunc      Warn if the result of an expression may be truncated
 --max-errors=number (default value : 0)
                      Stop after the given number of errors has been reached (by default: 100)
 --max-warnings=number (default value : 0)
                      Stop after the given number of warnings has been reached (by default: 100)
 --model=string       model name that will be generated. By default, all the model defined in the description file are generated.
 --templates=string   Template directory
Handled extension:
 .hadl                a gadl file that can be parsed with an LL1 grammar
\end{verbatim}

For each model described in the description, a directory is created containing the source code of the simulator. Consider for example the description file \texttt{avr.hadl} (included in \gadl\ sources) that contains the model \texttt{AT90CAN128}. The following command:

\begin{verbatim}
$ gadl ./avr.hadl
\end{verbatim}
create the directory  \texttt{./AT90CAN128}, which contains simulator sources. Most important files are:

\begin{verbatim}
Makefile           #file to generate the simulator (requires GNU Make)
arch.h             #main class of the simulator (arch)
arch.i             #C++/Python wrapper file (Swig)
format_all.dot     #describe the tree of the binary description of the instruction set (graphViz)
instConstruct.cpp  #instruction constructor (big file :-/)
instDecoder.cpp    #decode file (related to 'format')
instExec.cpp       #code executed by instructions (related to 'behavior')
instMnemo.cpp      #mnemonic of instructions (related to 'syntax')
instruction.h      #instruction classes
instruction.log    #debug information about the binary part
main.cpp           #simple scenario for the standalone simulator (not used with Python)
main_gdb.cpp       #simulate a gdb-server to use with GNU GDB (not used with Python)
storage.h          #memory model
types.h            #internal data types
\end{verbatim}

\subsection{Compilation options}
\label{sec:cflags}
Different compilation flags are available to modify the simulator behavior:
Main compilation parameters (\texttt{CFLAGS} in \emph{Makefile}) are:
\begin{description}
\item[HOST\_IS\_LITTLE\_ENDIAN] host is \emph{little endian}. This information is required by the \emph{storage} module.
\item[HOST\_IS\_BIG\_ENDIAN] host is \emph{little endian}. This information is required by the \emph{storage} module.
\item[INST\_DECODER\_CACHE\_STATS] This option allows to record memory instruction cache stats (nb of accesses, ratio hit/miss). The simulator speed is slightly altered.
\item[GADL\_NO\_ACTION] This option greatly accelerates the simulation speed. The \emph{action} are used to associate the execution of a piece of code to a memory access (read, write). Actions are used for peripherals description (\texttt{when} keyword). See section \ref{sec:whenReadWrite}.
\item[DEBUG\_MNEMO] To debug syntax part. the internal name of the instruction is printed with the mnemonic.
\item[DEBUG\_STORAGE\_ACCESS] To debug. Each memory access (read or write) is written on stdout. The simulator speed is altered
\item[DO\_NOT\_USE\_INTERNAL\_INSTRUCTION\_CACHE] To debug. Does not use the internal instruction cache associated to the decoder.
\item[DEBUG\_STORAGE\_ADDRESS\_RANGE] To debug. For each memory access, the simulator verifies that the address is in the correct range. Without this option, if an address is our of the memory chunk, a segfault should happen!
\end{description}

The \texttt{Makefile} should be used to generate the simulator (you should have a look to this documented file). Main targets are:
\begin{description}
\item[make standalone] generates a \emph{standalone} simulator. The simulation scenario is contained in the file \texttt{main.cpp};
\item[make python] generates a \emph{python library} of the simulator, see \ref{sec:python}. An example of how to use the python API is included with the AVR example.
\item[make gdb] generates a \emph{gdb server} simulator. This simulator can be used interactively used with gdb, using a UNIX socket. It has been tested successfully with Eclipse and the AVR module. This target works on POSIX hosts (Linux, Mac OS X).
\item[make clean] remove generated files.
\end{description}


\section{Using the simulator}
The API is available both for $C$/$C++$ and Python (through the SWIG wrapper).

The main $C++$ object is \texttt{arch} that should be used to have an instance of the processor.

Following sub-sections give main functions to deal with the simulator.
\subsection{System Observation}

\paragraph{\texttt{void decoderStats()}}
The method displays information on the internal decoder cache (used to speed up simulation). This command is valid only if the cache is used to decode (no option \texttt{DO\_NOT\_USE\_INTERNAL\_INSTRUCTION\_CACHE}) and if statistics are allowed (\texttt{INST\_DECODER\_CACHE\_STATS}), see section \ref{sec:cflags}:
\begin{verbatim}
Internal decoder instruction cache Ratio
    12128 accesses.
    Miss : 1217
    Hit : 10911
    Hit Ratio : 0.899654
    cache contains : 524 instructions
    (capacity is 1024 instructions)
    cache use ratio : 0.511719
\end{verbatim}

\paragraph{\texttt{unsigned int const getNBCycles()}} give the number of cycles since the beginning of the simulation. This is not used in the ISS mode (Instruction Set Simulation).
\paragraph{\texttt{unsigned int const getNBInstructions()}} give the number of instruction executed since the beginning of the simulation.
\paragraph{\texttt{string disassemble(const unsigned int ipStart, const int nbInst, bool verbose)}} disassemble code from address
\texttt{ipStart}, for \texttt{nbInst} instructions. If \texttt{verbose} is \texttt{true}, The decoding address and the instruction binary code are displayed.

\subsection{Execution}
\paragraph{\texttt{void reset()}}
At this date, it only reset the program counter.

\paragraph{\texttt{void execInst(const unsigned int nb)}} Execute \emph{a maximum of} \texttt{nb} instructions, either in CAS (Cycle Accurate Simulation) or ISS (Instruciton Set Simulation). If breakpoints are used, it should stop before.

\paragraph{\texttt{void runUntil(const unsigned int addr, const unsigned int max)}} simulate until the program counter is  \texttt{addr}. The simulator executes \emph{a maximum of} \texttt{max} instructions. If breakpoints are available, they are more efficient (no test at each clock cycle).

\paragraph{\texttt{bool readCodeFile(const char *filename, const bool verbose = false)}} read a binary code file and put it into the memory. Valid formats are: SRecord, Intel Hex or .elf, voir section \ref{sec:elf}). An error is generated if the file format is not recognized. Only memory section with \texttt{program} in the \h\ description of the memory can get program code (see section \ref{sec:mem_program}).

\subsubsection{Breakpoints}
Using breakpoints requires the \emph{actions} (execution of a piece of code related to a memory access (read, write or execute). Compilation flag \texttt{USE\_INTERACTIVE\_SIMULATION} should be set, and \texttt{GADL\_NO\_ACTION} should \emph{not} be set. see section \ref{sec:cflags}.
\paragraph{\texttt{void addBreakpoint(const unsigned int addr)}} Add a breakpoint for interactive simulation. An error is generated if there is already a breakpoint.
\paragraph{\texttt{void addBreakpoint(const char *symbolName)}} Add a breakpoint using the symbol name instead of the address (if symbols are available, with ELF files).
\paragraph{\texttt{void removeAllBreakpoints()}} 
\paragraph{\texttt{void removeBreakpoint(const unsigned int addr)}} Remove breakpoint at the specified address. An error is generated if there is not breakpoint.
\paragraph{\texttt{void removeBreakpoint(const char *symbolName)}} remove a breakpoint using symbol name instead of address.

\paragraph{\texttt{storage * getProgramChunk(const unsigned int address)}} Allow to get the memory chunk related to an address. This allows to read or write data in memory (see \texttt{storage.h} for further details).
\paragraph{\texttt{void setProgramCounter(u32 val)}} Define the value of the program counter (independently of the name used in the description (IP, PC,  \ldots).

\paragraph{\texttt{u32 programCounter()}} Read the value of the program counter (independently of the name used in the description (IP, PC,  \ldots).

\subsection{Symbols}
Using symbols instead of memory addresses is available only for .ELF files (see section \ref{sec:elf}).

\paragraph{\texttt{bool getSymbolObjectAddress(const char *symbolName, u32 \&v\_addr, u32 \&size)}} Search for the virtual address and size of the symbol name. It returns true if a symbol have been found, and updates \texttt{v\_addr} and \texttt{size}
\paragraph{\texttt{bool getFunctionName(const char *symbolName, u32 \&v\_addr)}} Search for the virtual address and size of the symbol name. It returns true if a symbol have been found, and updates \texttt{v\_addr}

\paragraph{\texttt{u32 getPhysicalAddress(const u32 v\_addr, bool \&found)}}  allows to get the physical address from the virtual one. \texttt{found} is set to \texttt{false} if the value is not in any available range.

it is easier to pool the common description of instructions

\subsection{Using Python}
\label{sec:python}
The main advantage of using Python to run scenarios is that it is no more necessary to recompile the simulator for each new scenario. When compiling the harmless simulator, a wrapper is generated with swig and a python module is built. In Python, it is then possible to call any C methods defined in the first part of this section.

For example, consider the AVR simulator provided in the examples directory. The model is called \texttt{AT90CAN128}, which is the name of the module to call. The following script calculates the number of instructions between 2 access to the function \texttt{function\_of\_task\_startTask}, 10 times:

\lstset{language=Python}
\begin{lstlisting}
#!/usr/bin/python
from AT90CAN128 import arch

f=arch()
f.readCodeFile("../trampoline.elf")
f.addBreakpoint("function_of_task_startTask")
nbInst = 0
f.printRegs()
for i in range(10):
    f.execInst(10000000) #run until breakpoint
    print f.getNBInstructions() - nbInst,  " insts between 2 breakpoints"
    nbInst = f.getNBInstructions()
f.printRegs()
f.decoderStats()
\end{lstlisting}
\lstset{language=Harmless}

%\part{Description du jeu d'instruction} 
%!TEX root = ./main.tex
%!TEX encoding = UTF-8 Unicode
\chapter{Modeling the Instruction Set}
\section{Modeling using 3 views}
\label{sec:modelisationArborescente}
\subsection{Tree modeling approach}
\harmless\ uses 3 views to describe the instruction set of a processor:
\begin{itemize}
\item the binary view (\emph{format}) is used to describe the binary format of instruction to enable the decoding stage of instructions
\item the behavioral view (\emph{behavior}) allows to describe the behavior of instructions to simulate instructions.
\item the syntax view (\emph{syntax}) allows to describe the mnemonic of instruction to generate the disassembler.
\end{itemize}

Each view is a set of tree where each node describes a portion of \emph{format}, \emph{behavior} or \emph{syntax}. The description of a node is as follows:
\begin{verbatim}
<kind> <name> <kind_options>
  <kind_body>
end <kind>
\end{verbatim}
where \texttt{<kind>} can be \texttt{format}, \texttt{behavior} or \texttt{syntax}. By default, a node aggregates the sub nodes which are defined in the body. However, using the keyword \texttt{select}, a node may be chosen among others in the description of each trees.

This tree approach aims to factor up similar items (for the syntax, behavior or binary format). In each view, an instruction is represented by a branch in a tree. Instructions that share certain characteristics in a view share the same nodes.

\subsection{Instruction identification}
\label{sec:signature}
A node can have one or more \emph{tags}. A tag is represented in  \harmless\ using the \texttt{\#} character, followed by an alphanumerical string.
The \emph{set} of \emph{tags} along a branch of a tree is the unique identifier of an instruction and is called the \texttt{Instruction Identification.} The identification is a \emph{set}, which means that there is no order and no counting of the number of occurrence of each tags.

In some cases, it is useful to \emph{mark} tags, because the same node can be used in different contexts. For instance, when reading 2 source registers and one destination register, we use \emph{extended-tags}. An \emph{extended-tag} is represented using the \texttt{@} character, followed by an alphanumerical character string. As a consequence, each tags in the subtree are modified by adding the \emph{extended-tag} as a suffix (without the \texttt{@} character).

This mechanism is used to link instruction between the different views (format, syntax and behavior).

\subsection{An example}
\label{exempleSignature}
More detailled examples are given in the chapters associated to each views. The goal here is to show how is described the tree structure.
Consider for example the following (simplified) code:

\begin{lstlisting}
behavior readGPR #read8
  ...
end behavior

behavior writeGPR #write8
  ...
end behavior

-- instruction that operate on 2 source registers
behavior twoRegsOp
  -- read the 2 source registers
  readGPR@src1
  readGPR@src2
  select
    case #ADC  ...
    case #ADD  ...
    case #SUB  ...
  end select
  writeGPR
end behavior
\end{lstlisting}
This example describe 3 \texttt{behavior} nodes (non terminal):  \texttt{readGPR}, \texttt{writeGPR}, and \texttt{twoRegsOp}. The node \texttt{twoRegsOp} is the root, because it is not called by any other node. 

\begin{figure}		%% Small Example
  \begin{center}
    \includegraphics[width=0.8 \linewidth]{../common/images/instBase.pdf}
    \caption{Tree representation of the simple example.}
    \label{fig:instBase}
  \end{center}
\end{figure}

It is graphically represented in figure \ref{fig:instBase}. Each node is represented by name and by a vertical bar. To the right of the vertical bar, the various "elements" (selection, call for another node with or without \emph{extended-tag}) are shown \emph{sequentially}. The circle represent the structure of selection (\texttt{select}). Tags are represented in rectangles, and extended-tegs are represented in the calling node.

The advantage of using one extended-tag here lies in the fact that the \texttt{readGPR} node is called in 2 different contexts: one for each source register. As the extended-tag is added to each tags in the subtree, there are differentiated: \texttt{\#read8src1} and \texttt{\#read8src2}.

So in this example, the ADD instruction as a set of tags:  \texttt{\#read8src1}, \texttt{\#read8src2}, \texttt{\#ADD} and \texttt{\#write8}.
Similarly, the SUB instruction as a set of tags: \texttt{\#read8src1}, \texttt{\#read8src2}, \texttt{\#SUB} and \texttt{\#write8}. These set of tags represent the \emph{instruction identification}.

In the internal representation, each instruction is modeled using a $C++$ class, where the name of the class is here  \texttt{cpu\_ADD\_read8src1\_read8src2\_write8}. To define the internal name of an instruction, the name of the model (here \texttt{cpu} is concatenated with the set of tags in the alphabetical order, separated by the '\texttt{\_}' character.

%\subsection{Intérêt de la structure selon 3 vues}
%TODO 

%!TEX root = ./main.tex
%!TEX encoding = UTF-8 Unicode
\chapter{Fundamentals of language}
\section{Lexical Elements}
This section lists all lexical elements which are used by \harmless. Spaces and indentations are ignored by the lexical analyzer (as in C): indentation has no influence on the language, although it facilitates the reading!
As many instructions look very much like the $C$ language, the $;$ character is understood as a space.

\subsection{Comments}
\label{sec:commentaire}
Comments use the same syntax as in VHDL. They begin with \texttt{--} and ends at the end of the line.

\begin{lstlisting}
--  This is a comment
\end{lstlisting}

\subsection{Character String}
\label{sec:chaines}
A string is represented using double quotes \texttt{"} (like in $C$):
\begin{lstlisting}
"This is a character string"
\end{lstlisting}
Carriage returns are ignored.

\subsection{Integers}
\label{sec:nombres}
Integers can be written with different bases: binary, decimal, octal and hexadecimal, preceding the number by respectively \texttt{\bs b}, \texttt{\bs d}, \texttt{\bs o} and \texttt{\bs x}. Decimal format is used by default:
\begin{lstlisting}
38    -- 38 (decimal)
\d12  -- 12 (decimal -> default)
\b100 -- 4 (binary)
\o70  -- 56 (octal)
\x2F  -- 47 (hexadecimal)
\end{lstlisting}

The \texttt{s} character is used as a suffix for \emph{signed integers}.
\begin{lstlisting}
38  -- unsigned integer -> 6 bits
38s -- signed integer   -> 7 bits
\end{lstlisting}

To help readability, the \texttt{\_} character can be added everywhere in the integer definition, it will be deleted by \harmless\ during lexical analysis:
\begin{lstlisting}
\b1001_1111 -- 9F hexa
\end{lstlisting}

Finally, it is possible to add the suffix \texttt{kb} ($2^{10}$=1024 bytes) and \texttt{mb} ($2^{20}$= 1048576 bytes). It is useful for address space definition:
\begin{lstlisting}
128kb -- equivalent to 131 072
4mb   -- equivalent to 4194304
\end{lstlisting}

\subsection{Mask}
\label{masque}
A mask allows to express a set value depending on the value of different bits. The masks are only values coded as binary:
\begin{lstlisting}
\m11-0 -- correspond to 1110 or 1100
\end{lstlisting}
A mask is prefixed by \texttt{\bs m}. The binary number is expressed by \texttt{0}, \texttt{1} and \texttt{-}, the latter indicating that the value of the corresponding bit is not useful. 

Masking operations are mainly used for decoding instructions (see Chapter \ref{chap:format}).

\subsection{Floats} 
There is currently no support for floating point numbers in Harmless.

\subsection{Keywords}
\label{keywords}
Reserved keywords for the language are:
\begin{lstlisting}
model, port, device, architecture, write, shared, behavior, format, select, error, warning, component, void, every, memory, width, address, type, RAM, ROM, register, stride, read, program, counter, pipeline, stage, init, run, as, machine, BPU, bypass, release, in, maps, to, default, instruction, fetch, debug, big, little, endian, except, do, out, when, on, field, nop, slice, case, is, others, signed, or, syntax, switch, number, octal, decimal, hexadecimal, binary, suffix, prefix, return, print, if, then, elseif, else, loop, while, end, true, false, ror, rol, cat, timing, decode, size, jumpTaken, add, cycle, use, interrupt
\end{lstlisting}

\subsection{Delimiters}
Delimiters are mostly used for expressions and assignments:
\begin{lstlisting}
: .. . , {   } [ ] :=  ( ) !  ~  * /  %  + - <<  >>  < >  <=   >=  
=  !=  &  ^  |  &&  ^^  ||;
\end{lstlisting}

\subsection{Tags}
\label{sec:etiquettes}
Tags are used to enable a correspondence between different views of an instruction (behavior, binary and syntax). An instruction is then modeled as a set of tags, which form its \emph{signature} (see chapter \ref{sec:signature}).

\emph{Tags} begins with character \texttt{\#}, followed by alphanumerical characters (\texttt{a..z}, \texttt{A..Z}, \texttt{0..9}, \texttt{\_}).

In few cases, we use an \emph{extended tag}, which is a tag that will be added for each node called. An \emph{extended tag} begins with character \texttt{\@}, followed by alphanumerical characters (\texttt{a..z}, \texttt{A..Z}, \texttt{0..9}, \texttt{\_}).

\subsection{Identifier}
\label{sec:identifiant}
An identifier is an alphanumerical characters string (\texttt{a..z}, \texttt{A..Z}, \texttt{0..9}, \texttt{\_}) where the first letter is not a number. It cannot be a keyword, nor a data type (\texttt{u} or \texttt{s} followed by a number), see section \ref{sec:TypeDonnees}.

\section{Data types}
\label{sec:TypeDonnees}
At this date, \harmless\ can only handle integer data types. Data may be signed or unsigned, and size is defined at the bit level. Data types are defined using  \texttt{s} (signed) or \texttt{u} (unsigned) characters, followed by a number which define the size of the data in bits:
\begin{lstlisting}
u17 val1 --  val1 is an unsigned 17 bits value
s9 val2  --  val2 is a signed 9 bits value
u1 bool  -- u1 is understood as a boolean value.
\end{lstlisting}
Internal implementation limits sizes to 64 bits at this date, because these types are directly mapped on $C$ types.

An immediat value has the minimal size required to be coded:
\begin{lstlisting}
38  -- unsigned integer -> 6 bits -> u6
38s -- signed integer -> 7 bits -> s7
-1s -- signed integer -> 2 bits -> s2
\end{lstlisting}

\section{Cast}

The \emph{cast} allows to change the type of a value. A sign extension is performed when casting \emph{to a signed type} (and \emph{only} to a sign type. Examples:
\begin{lstlisting}
(s8)(-1s)  -- cast a s2 (signed with 2 bits) to a s8: result is 0xFF
(u32)(-1s) -- cast a s2 to a u32 (no sign extension!): result is 0x03
(u32)((s32)(-1s)) -- cast a s2 to a s32 (sign extension) and to u32
\end{lstlisting}


%\begin{verbatim}
\section{Expressions}
\label{sec:expressions}
Expressions are largely inspired by the $C$ language, with some extensions for bit manipulations. We found the following $C$ operators: $($, $)$, $!$, $\sim$, $*$, $/$, $\%$,$+$, $-$, $<<$, $>>$, $<$, $>$, $<=$, $>=$, $=$, $!=$, $\&$, $\wedge$, $|$, $\&\&$, $||$.

Note, however, even if these expressions are slightly the same as in $C$, there are differences on the size of values returned. See \ref{sec: typeExp}

Expression enhancements are about type casts, rotations, concatenation and access to bit fields, these operation are not available in $C$.

\subsection{Expressions Priorities}
All expressions sorted by priority are defined in Table \ref{tab:exp}.

\begin{table}[!h]
\begin{center}
\begin{tabularx}{\columnwidth}{|c|c|X|}
\hline
\bf expression & \bf priority & \bf use  \\  \hline
idf & 1 & variable, register, memory access. Call to a component's method ... \\ \hline
idf[index] & 1 & access to a tabular value \\ \hline
nombre & 1 & numerical value (signed or not). \\ \hline
(exp) & 1 & parentheses\\ \hline
(type)(exp) & 1 & cast expression 'exp'. Unlike the $C$, parentheses are required, in order to eliminate This expression ambiguity \\ \hline
\{field\} & 2 & Access to a bit field. See \ref{sec:field} \\ \hline
 ! & 3 &  Logical not: returns \emph{true} or \emph{false} \\ \hline
$\sim$  & 3 & bitwise not \\ \hline
-  & 3 & unary minus \\ \hline
* / \%  & 4 & multiplication, division, modulo \\ \hline
+ -  & 5 & Addition, subtraction \\ \hline
$<<$ $>>$ & 6 & left and right shift (as in $C$)\\ \hline
ror rol & 6 & right and left rotation: \texttt{exp rol 3} rotate bits of \emph{exp} 3 bits to the left \\ \hline
$<$ $>$ $<=$ $>=$ & 7 & logical comparison \\ \hline
$=$ $!=$  & 7 & equal logic \\ \hline
$\&$  & 8 & bitwise and \\ \hline
$\wedge$  & 9 & bitwise xor \\ \hline
$|$   & 10 & bitwise or \\ \hline
$\&\&$  & 11 & logical and \\ \hline
$||$   & 12 & logical or \\ \hline
cat   & 13 & concatenation of expressions \\ \hline
\end{tabularx}
\caption{Expressions evaluation priorities in \harmless (1 being the highest priority)}
\label{tab:exp}
\end{center}
\end{table}

%TODO: exemple pour ror/rol, cast et cat.

\subsection{Result type expressions}
\label{sec:typeExp}
The type of an expression is strong. It is not based solely on the size of the input data, but also on operations that are performed on it. For example:
\begin{itemize}
\item adding 2 numbers of $n$ and $m$ bits returns a result of $max(n,m)+1$ bits;
\item multiplication of 2 numbers of $n$ et $m$ bits returns a result of $n+m$ bits;
\item shifting $d$ bits left a value of $n$ bits returns a result of $n+d$ bits;
\item shifting $d$ bits left a value of $n$ bits returns a result of $n-d$ bits;;
\end{itemize}
%TODO: pas fini.

\subsection{Bitfield access}
\label{sec:field}
The access to a field is using braces. The definition of a field by setting the most significant bit first, followed by '\texttt{..}' and the lowest bit. The second value must be lower than the first.

\emph{In \harmless, the LSB has always the index 0}. The MSB has index 'size of data' - 1. 
This is notably not the case with some manufacturers documentation indicating the bit 0 as the most significant bit (PowerPC, for example).If a single bit is used, the second part is not required. Several fields can be defined, separated by commas:
\begin{lstlisting}
u8 val1 --  val1 is unsigned 8 bits
 -- val2 gets the 4 lowest significant bits of val1
u4 val2 := val1{3..0}
--  val3 gets bit 4, concatenated with the 2 lowest significant bits of val1.
u3 val3 := val1{4, 1..0}
u8 val4 := 3;
-- we can use expressions to define a field.
u2 val5 := val1{val4+1..val4}
\end{lstlisting}

You can also use an expression to define an element of a field, but this expression must return an \emph{unsigned} value. Expressions can not be allowed in certain cases (format definition), because \harmless is not always able to extract the size of the field.

\section{Instructions}
language instructions are used in a limited area for implementation: in a \emph{do}..\emph{end do} when defining behavior or in the definition of a component, ...

There are no instruction for loops at this date, even if a statement will be added in the short term (the instruction type \emph{store multiple word} of the PowerPC, for example).

Following statement are provided:
\subsection{Assignment}
assignment use the operator \texttt{:=}, to avoid ambiguity with a comparison: \texttt{<variable> :=  <expression>}:
\begin{lstlisting}
val2 := val1{3..0} -- val2 gets the 4 lowest significant bits of val1
\end{lstlisting}

It is allowed to use bitfields in the left part of an assignment (see \ref{sec:field}):
\begin{lstlisting}
u8 val2;
-- assign the 4 most significant bits of val2
-- the 4 loewt significant bits are not modified.
val2{7..4} := val1{3..0}
\end{lstlisting}

\subsection{Conditional statement}
The conditional statement has the form: \texttt{if <expresssion> then <implementation> [elseif <implementation>] [else <implementation>] end if}
\begin{lstlisting}
u16 newPC;
if CCR{bitIndex} = 0 then 
  newPC := (u16)((s16)(PC)+k) 
else 
  newPC := (u16)(PC)
end if 
\end{lstlisting}

\subsection{Loops}
Loops have the form: \texttt{loop <guard> while <condition> do <implementation> end loop}. \texttt{Guard} is the maximal number of iterations allowed, due to prevent infinite loops. The algorithm used by the simulator is:
\begin{lstlisting}
u64 loop = 0;
while(loop < guard && condition)
  loop = loop + 1;
  <implementation>
if(loop == guard)
  -- send runtime error.
\end{lstlisting}

\emph{Loops are allowed in an \harmless\ description only to generate an instruction set simulator. There are many restrictions with CAS.}

Loops may be used to model instructions when the algorithm is based on a loop, but the hardware implementation does not need such mechanism. This is the case for instance of the ARM instruction CLZ (\emph{Count Leading Zeroes} that counts the number of zeros, from the MSB. For this instruction, we may use the following code:
\begin{lstlisting}
u6 clz(u32 value)
{
  u1 found := false
  u6 currentBit := 32
  loop 32 
  while (!found && currentBit != 0) do
    if value{currentBit-1} then 
      found := true
    else
      currentBit := currentBit - 1
    end if
  end loop
  return 32 - currentBit
}
\end{lstlisting}
The guard limits the loop to 32 iterations.

\emph{In the case of instructions where the implementation is based on a loop (\emph{Load/Store Multiple Word} for PowerPC and ARM instruction sets), the generated CAS does not take into account multiple access to methods (access only one time).}

\subsection{Return statement}
This statement can return a comma separated list of values within a method component(section \ref{sec:component}), the same approach as C. This statement is not always available (for example in the implementation of a \emph{behavior}).

The form is: \texttt{return <expression1>,<expression2>, \ldots}:
\begin{lstlisting}
return val1,val2
\end{lstlisting}

\subsection{Nop statement}
This instruction can inhibit xx next instructions. It is a feature found on some processors such as Atmel AVR. This instruction is available on the implementation part of a  \texttt{behavior}.

The form is: \texttt{nop <expression> instruction}
\begin{lstlisting}
-- the next instruction won't be executed.
nop 1 instruction
\end{lstlisting}

\subsection{Error statement}
\label{sec:instError}
These instructions allow to generate run-time error or a warning on an abnormal state of the simulator.

Some devices can be partially described: a \emph{timer/counter} with only the model of the \emph{timer} part, prohibited combinations from documentation, \ldots. These cases can be detected by the simulator. 

The form is: \texttt{error <character string>} or \texttt{warning <character string>}. 
At this date, only the error string is printed on \texttt{stderr} at run-time.

\begin{lstlisting}
warning "error message"
\end{lstlisting}

At run-time, the following message will be printed on \texttt{stderr}: 
\texttt{RUNTIME WARNING at file '/Users/mik/../avr.hadl', line 54:30. Message is "error message"}

\subsection{Display statement}
This instruction allows you to write a string on the error output (stderr). This is particularly useful to model peripherals: \\
\texttt{print (<expression>|<characterStr>)[,(<expression>|<characterStr>)][, ...} 

For a serial port: 
\begin{lstlisting}
print UDR0
\end{lstlisting}
Will print the value of the register \texttt{UDR0} on stderr. The value will be interpretated has in $C$ (ASCII for a value of 8 bits or less), numerical value in other cases (hex).

For a GPIO:
\begin{lstlisting}
print "port A: ",PORTA,"\n"
\end{lstlisting}

\subsection{Interrupt}
\label{keyword:interrupt}
Interrupt hardware management is described in chapter \ref{chap:interrupt}. The \texttt{interrupt} keyword allows to set an interrupt. 
The form is: \texttt{interrupt <unsigned number>}, where the unsigned number is the \texttt{id} of the interrupt.
\begin{lstlisting}
interrupt 5
\end{lstlisting}
This value will be available for the hardware interrupt handler.

\section{Organization of a description}
A description follows the general schema:
\begin{verbatim}
<modelDeclaration> 
repeat 
  while <inModel>; 
  while <default>; 
  while <component>; 
  while <pipeline>; 
  while <machine>; 
  while <architecture>; 
  while <format>; 
  while <behavior>; 
  while <syntax>; 
  while <timing>; 
  while <printNumberType>; 
end repeat;
\end{verbatim}
The structure \texttt{repeat..while..end repeat} indicates that it is possible to put each of these rules in any order as many times as needed (even 0).
The parts are:
\begin{itemize}
\item \texttt{modelDeclaration} identifies the model. It is possible to declare many models in the same description, see section \ref{sec:plusieursModeles};
\item \texttt{inModel} is used when describing many models, see section \ref{sec:plusieursModeles};
\item \texttt{default} allows to define default parameters. It is mandatory and defined in section \ref{sec:default};
\item \texttt{component}  describe an hardware \emph{component}, see section \ref{sec:composant};
\item \texttt{format} \texttt{behavior} and \texttt{syntax} are related to the instruction set description (3 views), see \ref{chap:format},  \ref{chap:behavior} and  \ref{chap:syntax};
\item \texttt{printNumberType} is used in syntax description, see \ref{chap:syntax};
\item \texttt{timing} is used in a temporal description, without taking into account the underlying micro-architecture;
\item \texttt{pipeline} \texttt{machine} and \texttt{architecture} are used for the description of the micro-architecture.
\end{itemize}
No order of the different part is required.

\subsection{Dealing with multiple description files}
\label{sec:plusieursFichiers}
The description of a processor can use multiple files, which are then used to generate a simulator. 
For example, the instruction set of the PowerPC can be described in a first file \texttt{powerpc\_instSet.hadl} which is then used in the description of different models (\texttt{5516}, \texttt{565}, \texttt{750}, \texttt{970}, ...):
\begin{lstlisting}
include "powerpc_instSet.hadl"
\end{lstlisting}
Currently, no verification is performed to detect cyclic inclusions.

%section "défaut."
\subsection{\emph{Default} section}
\label{sec:default}
The \texttt{default} section is mandatory and set some global settings.
\subsubsection{default size of instructions}
This parameter gives the basic size of instructions for decoding. For example, all of the PowerPC instructions are 32 bits wide, so the value should be set to 32, but with the HCS12, where instruction sizes are from 1 to 8 bytes, the value is 8:
\begin{lstlisting}
default {
  instruction := 8
}
\end{lstlisting}
This parameter is used in the decode phase.

\subsubsection{Endianness}
The endianness should be given (used when accessing memory):
\begin{lstlisting}
default {
  big endian
}
\end{lstlisting}
or \texttt{little endian} when using the lowest significant byte first.
\emph{At this date, it's not possible to change endianness dynamically.}
This parameter is mandatory.

\section{Description of several models in the same file}
\label{sec:plusieursModeles}
This approach is used to describe different variants of an architecture.
The declaration of the different models is:
\begin{lstlisting}
model mod1, mod2, mod3 
{
\end{lstlisting}
In this example, the file contains 3 models \texttt{mod1} to \texttt{mod3}. In the generation process, directories \texttt{mod1/} to \texttt{mod3/} will be created, each one having the sources of one simulator.

The description is common to all the models by default. To get a specific part, the following command should be used:
\begin{lstlisting}
-- the description between {} is valid
-- only for mod1 and mod2:
in mod1, mod2 {  bla bla bla }
\end{lstlisting}
the \texttt{*} is similar to 'every models', and the keyword \texttt{except} is used to remove particular models
\begin{lstlisting}
-- identical to the previous description.
in * except mod3 { 
\end{lstlisting}

The granularity of the \texttt{in} keyword is restricted to high level elemets: a whole component, a \texttt{behavior}, a \texttt{format}, \ldots

%!TEX root = ./main.tex
%!TEX encoding = UTF-8 Unicode
\chapter{Description of instructions binary code}
\label{chap:format}
The description of the instruction format is based on a tree structure, allowing up to pool the common format of each instruction. The objective here is to extract from the binary format of instruction, both the type of instruction, and the operands. In the generated simulator, this step takes place in the \emph{decoding} phase of instructions.

We will study in a first approach fixed size instruction set. The description will then be extended to variable size instruction sets in section \ref{sec:formatTailleVariable}.

\section{General architecture}
The description of the binary format will allow to decode the binary format of instructions. Initially, it is necessary to provide in section \texttt{default} the basic size of instructions:
\begin{lstlisting}
default {
    instruction := 16  -- default instruction size in bits
}

\end{lstlisting}
In this example for instance, the instruction size is 2 bytes or a multiple of 2 bytes for the variable-size instruction sets. For example, instructions sizes in the \texttt{HCS12} are from 1 to 8 bytes. In this case, you should specify \texttt{instruction := 8}.

The description of the format describes how to decode the binary word whose size is supplied.

The general structure of the description of format nodes is:
\begin{lstlisting}
format <name> [tag]
  <formatBody>
end format
\end{lstlisting}

\texttt{<formatBody>} is a sequence of elements:
\begin{itemize}
\item \emph{tags};
\item call to another format
\item select structure, using the \texttt{select} keyword. See section \ref{sec:formatSelect};
\item definition of operands. See section \ref{sec:operandeFormat}.
\end{itemize}

\subsection{Select structure}
\label{sec:formatSelect}
The \texttt{select} statement allow to differentiate different branches of the tree, like in the description of each view.

The select statement acts on a bit field of the binary format of the instruction:
\begin{lstlisting}
  select slice <field>
    case <masque> is <formatBody>
    case .. 
  end select
\end{lstlisting}

\subsubsection{\texttt{field} part}
The \texttt{field} part indicate the binary part of the instruction code that will be used to differentiate instructions. For example, let's consider instructions \texttt{ADDC} and \texttt{ADD} of AVR, where the binary representation is given in figure \ref{fig:selectFormat1}.

\begin{figure}[h]		%% select sur ADD et ADDC
  \begin{center}
    \includegraphics[width=0.8 \linewidth]{../common/images/selectFormat1.pdf}
    \caption{Binary code of instructions \texttt{ADDC} and \texttt{ADD} of Atmel AVR® (code size is 16 bits).}
    \label{fig:selectFormat1}
  \end{center}
\end{figure}

In this case,the following construction can be used:
\begin{lstlisting}
  select slice {12}
    case 0 is ... -- ADC
    case 1 is ... -- ADD
  end select
\end{lstlisting}

Indeed, only bit 12 distinguishes the 2 instructions. The \texttt{<field>} part is written in the same way than a bitfield of a variable (section \ref{sec:field}), but only numerical values can be used (no expressions):
\begin{lstlisting}
  select slice {12..10, 3..2} -- select bits 12 to 10 and 3 to 2 (5 bits selected)
\end{lstlisting}
The  \texttt{<field>} part size should be guessed statically.

\subsubsection{\texttt{mask} part}
The \texttt{mask} part operates on bit fields taken from the \texttt{field} part. It could not have a higher size than the \texttt{field} one (an error is generated during generation of the simulator).

The \texttt{mask} part can be either a numerical value (as in the previous example) or a binary mask (see section \ref{masque}). When using a binary mask, charater \texttt{-} can either be understood as a \texttt{0} or \texttt{1}. Using the example of figure \ref{fig:selectFormat1}, if we want to pool the common part of the 2 instructions:
\begin{lstlisting}
  select slice {15..10}
    case \m000-11 is .. -- both ADD and ADDC
    case ...
  end select
\end{lstlisting}
In this case, a selection is done on the whole \emph{opcode} (binary code except the operands). The first \texttt{case} will be taken only if bits \texttt{15..10} are \texttt{000011} (\texttt{ADD}) or \texttt{000111} (\texttt{ADDC}).

The \texttt{or} operator allow to select different mask for one case:
\begin{lstlisting}
select slice {7..0}
    case \x1B or \x1A or \x19 is ...
    ..
\end{lstlisting}
This example, extracted from the description of the Freescale \emph{HCS12}, allow to pool different cases, although their correspondence at the binary level is not possible.

Eventually, all the other case may be pooled using the \texttt{others} keyword.
\begin{lstlisting}
  select slice {2..0}
    case \m110 is ...
    case \m111 is ...
    others     is ...
  end select
\end{lstlisting}
In this example, the last choice will be taken for all formats that not match \texttt{11-}.

The \texttt{others} keyword can only be used one time in a \texttt{select} statement, and it should be the last case.

\subsubsection{\texttt{<formatBody>} part}
The \texttt{<formatBody>} is the body of a \emph{format} node.

\subsection{Extracting operands}
\label{sec:operandeFormat}
The extraction of operands use the \texttt{slice} keyword, with a bitfield. Given the 2 instructions \texttt{ADD} and \texttt{ADDC} in figure \ref{fig:selectFormat1}:
\begin{lstlisting}
  r := slice{9, 4..0} -- size: 5 bits
  d := slice{8..4}    -- size: 5 bits
\end{lstlisting}
Size of operands is computed statically. In other views (\emph{behavior} and \emph{syntax}), operands values can be retrieved as constant. In that case, the \texttt{field} keyword is used:
\begin{lstlisting}
  field u5 r
  field u5 d
\end{lstlisting}

Some operands need to be declared as signed values (for a branch for example). This is done using the \texttt{signed} keyword:
\begin{lstlisting}
  k := signed slice{11..0}
\end{lstlisting}
\texttt{k} type is \texttt{s12}.

Eventually, it may be useful to operates directly on the operands. A shift operator is provided (limited to a numerical value):

\begin{lstlisting}
  RdIndex := slice{7..4} << 1
\end{lstlisting}
In that case, \texttt{RdIndex} type is \texttt{u5}.

\section{Example}

This example is based on a part of the XGate instruction set. XGate is a co-processor integrated in the \emph{HCS12X}. It is based on a RISC architecture, with a 16 bits, fixed sized, instruction set. 16 GPR (8 bits) are provided.

\subsection{Description of a part of the instruction set of XGate}
Figure \ref{fig:shiftAndTriadicInstFormat} gives the binary format of shift and triadic instructions.

\begin{figure}[h]		%% select sur ADD et ADDC
  \begin{center}
    \includegraphics[width=0.95 \linewidth]{../common/images/shiftAndTriadicInstFormat.png}
    \caption{Binary code of shift and triadic instructions of XGate® \emph{Source Freescale}}
    \label{fig:shiftAndTriadicInstFormat}
  \end{center}
\end{figure}

We can directly note on this example that the bit 12 serves to differentiate the 2 types of instructions. More generally for the whole instruction set, the 5 most significant bits allow to identify instruction families (\emph{opcode}):
\begin{lstlisting}
format inst
  select slice {15..11}
    case \b00001 is shiftInstructionImm
    case \b00010 is logicalTriadic
  end select
end format
\end{lstlisting}
For shift instructions, the \texttt{shiftInstructionImm} format node is used:
\begin{lstlisting}[firstnumber=7]
format shiftInstructionImm #IMM
  rdIndex := slice{10..8}
  imm4 := slice{7..4}
  select slice {3..0}
    case \b1001 is #ASR
    case \b1010 is #CSL
    case \b1011 is #CSR
    case \b1100 is #LSL
    case \b1101 is #LSR
    case \b1110 is #ROL
    case \b1111 is #ROR
  end select
end format
\end{lstlisting}
The instructions identification is \texttt{\#IMM \#ASR} for instruction \texttt{ASR}, \texttt{\#IMM \#CSL} for \texttt{CSL}, \ldots

For triadic instructions, the format node is:
\begin{lstlisting}[firstnumber=20]
format logicalTriadic #Triadic
  rdIndex := slice{10..8}
  rs1Index := slice{7..5}
  rs2Index := slice{4..2}
  select slice {1..0}
    case \b00 is #AND
    case \b10 is #OR
    case \b11 is #XNOR
  end select
end format
\end{lstlisting}
Thus, we have described here the binary format of these 10 (simple) instructions in 29 lines.

\subsection{Decoder generation}
The decoder may be generated, even if the other views (syntax and behavior) are not written. In that case, the following options should be given:
\begin{verbatim}
$ gadl --no-behavior --no-deasm test.hadl
\end{verbatim}
This allow to detect both syntax and semantic error in the format description. 

Moreover, it detects the orthogonality of the instruction set (a binary code can be associated to only one instruction). This verification may require computation time (few seconds for 1500 instructions) and can be removed when the binary description is done using option \texttt{--no-check}.

\subsection{Output log file}

The generated decoder files are located in \texttt{instDecoder.h} and \texttt{instDecoder.cpp}. Some other files are generated for debugging phase

Files \texttt{format\_all.dot} and \texttt{format\_ref.dot} allow to display the format tree related to instructions, in GraphViz format. The first one display the whole format (including \emph{format} names \emph{select} part), see figure \ref{fig:formatAllTest}. 

\begin{figure}[h]		%% format All
  \begin{center}
    \includegraphics[width=\linewidth]{../common/images/format_all_test.pdf}
    \caption{Arbre généré à partir de la description du code binaire des instructions de décalage (shift) et les instruction triadiques (3 registres) sur la XGate.}
    \label{fig:formatAllTest}
  \end{center}
\end{figure}


This tree may become quickly difficult to display, that's why the second one only display \emph{tags} for each instruction, see figure \ref{fig:formatRefTest}. In this last figure, there are 2 distinct trees, because shift instruction and triadic instruction do not share anu common properties (no \emph{tags}).


\begin{figure}[h]		%% format Ref
  \begin{center}
    \includegraphics[width=\linewidth]{../common/images/format_ref_test.pdf}
    \caption{Trree generated from the description of shift and triadic instructions on the XGate instruction set. This tree is limited to tags.}
    \label{fig:formatRefTest}
  \end{center}
\end{figure}

Another file is \texttt{instruction.log} that embeds information about binary code of instructions:
\begin{itemize}
\item the path followed in the tree to get the instruction;
\item the \emph{instruction id}: Tags are ordered in alphabetical order;
\item the internal name used by instruction (debug in the generated \emph{C++} files)
\item the binary opcode (no date fields)
\item the name and size of data fields
\item the size of the instruction (word size is defined in section \emph{default});
\end{itemize}

For the triadic instruction \texttt{AND}, we get the following information:
\begin{verbatim}
inst 
-> select_format_0 
-> format_case_2 
-> logicalTriadic 
-> #Triadic 
-> select_format_11 
-> format_case_12 
-> #AND
	instruction id :#AND, #Triadic
	Internal name :test_AND_Triadic
	Binary coding :00010---------00
	data field(s) :rdIndex (3 bits)
	               rs1Index (3 bits)
	               rs2Index (3 bits)
	Instruction code size :1
\end{verbatim}

%TODO: debug -> fichier de log
%intérêt de l'archi avec les codes de l'ARM.
%indiquer que les performances du décodeur ne dépendent pas de la structure de l'arbre de description.

\section{variable size instruction set}
\label{sec:formatTailleVariable}
Variable size instruction set are described by adding binary words directly in the nodes.
In the following example, extracted from the \texttt{Freescale HCS12} instruction set (instruction word is 8 bits):
\begin{lstlisting}
format Instruction 
  select slice {7..0}
    case \x18 is inst_18 
    ....
  end select
end format

format inst_18 
  select slice +{7..0}
    case \m00010111 is #CBA       -- 17h
    ...
  end select
end format
\end{lstlisting}
In this example, the first node (\texttt{Instruction}) allows to defined the first word (as in fixed size instruction sets). Node \texttt{inst\_18} is called if this first word \texttt{0x18}. In format \texttt{inst\_18}, the \texttt{+} in the \texttt{select} part, line 9 implies that it is necessary to add a binary word to decode the instruction.
In this way, semantics of \texttt{+\{7..0\}} is: \emph{adding one word, and apply the \texttt{select} on the 8 bits of this new word.}

A \texttt{select} that have the following form:  \texttt{\{a..b\}\{c..d\}+\{e..f\}\{-\}\{g..h\}} show that 3 words are added for instructions de the current branch, and the \texttt{select} is applied on:
\begin{itemize}
\item bit field \texttt{a..b} of the previous word;
\item bit field \texttt{c..d} of the current word;
\item bit field\texttt{e..f} of the first word added;
\item the second added word is not used in the select structure;
\item bit field\texttt{g..h} of the third word added;
\end{itemize}
If there was no word previously added (there is only the current word), an error is generated.

If a new format is called in or after the \texttt{select}, le last added word becomes the current word.

This approach using relative access (adding words to previously added words) allow to describe easily instruction formats where the addressing mode is not always at the same place, as in the example figure \ref{fig:formatLongueurVariable}.

\begin{figure}		%% Small Example
  \begin{center}
    \includegraphics[width=0.8 \linewidth]{../common/images/formatLongueurVariable.pdf}
    \caption{Example of variable size instruction set, extracted form the documentation of HCS12. The addressing mode \texttt{xb} may be located at different places (source Freescale).}
    \label{fig:formatLongueurVariable}
  \end{center}
\end{figure}

%TODO: rajouter le fetch explicite!!!
%!TEX root = ./main.tex
%!TEX encoding = UTF-8 Unicode

\chapter{Description of instructions behavior}
\label{chap:behavior}
\section{Introduction}

In this chapter, we explain how instruction behavior are described. Many examples are extracted from descriptions of the Freescale HCS12 and its coprocessor XGate. The behavior of instructions is described in a hierarchical manner so as to share common behaviors of instructions.

The description of the behavior of instructions is based on objects called \texttt{component} in the language. A component models an hardware part of the processor like the register bank or the UAL. Obviously, we can describe the behavior of instructions without using any component (using the '+' operator of the language to model the behavior of the \texttt{add} instruction), but this approach is not recommended. Indeed, the temporal behavior of a micro-architecture requires the use of components as the concurrency is based on the components.

\section{Behavior view in \h}

The behavior view contains a set of behavior nodes, which are organized in a hierarchical approach (as in the format and syntax views): a behavior can call other elementary behaviors and the behavior view may contain many trees. For instance, an addressing mode may be modeled using a behavior, and will be used by many other behaviors. 

A behavior node is similar to a function that will call other functions (other behaviors, or component's method) to perform operations required to execute the instruction. The following listing shows how to describe a behavior node:
\begin{lstlisting}
behavior <name>(<parameterList>)
  <behaviorBody>
end behavior
\end{lstlisting}

{\tt <behaviorBody>} is a combination of elements:
\begin{itemize}
\item variable declaration(see section\ref{sec:behVar});
\item {\em field} declaration (see section\ref{sec:behField});
\item \emph{tags};
\item call to another behavior node (\ref{sec:behSubBeh});
\item select structure to distinguish different instructions, using keyword \texttt{select}. See section \ref{sec:behSelect};
\item \blocsdo\ that allow to use algorithms. Voir section \ref{sec:behDo}.
\end{itemize}

\subsection{Behavior's parameters}

A behavior (except root behaviors) may have 1 or more input/output parameters. A parameter have type, and an output argument uses the \texttt{out} keyword. Parameters handling is the same as the one for function call: 
\begin{lstlisting}
behavior shiftInstructionBehavior(out u16 rdValue, u16 source)
\end{lstlisting}
Here, 2 parameters are givent. The first one \texttt{rdValue} is an output parameter (its value will be set by \texttt{shiftInstructionBehavior} and will be available by the behavior that calls \texttt{shiftInstructionBehavior}). The second one \texttt{source} is an input parameter.

\subsection{Variable declaration}
\label{sec:behVar}

It is possible to declare local variables for a component as in section \ref{sec:TypeDonnees}.

\subsection{Instruction Field reference}
\label{sec:behField}

Instructions format fields that contains numerical data may be declared in a component in order to use their value. The keyword \texttt{field} is used:
\begin{lstlisting}
  field u3 regIndex;
\end{lstlisting}
Here, \texttt{regIndex} is a field that have been extracted from the format view (see section \ref{sec:operandeFormat}). \h\ checks the type consistency (size and sign) and displays an error if types differs. A \texttt{field} is obviously always a constant.

\subsection{Calling another behavior}
\label{sec:behSubBeh}

Calling another component is done using the name of the behavior with theirs required parameters. A call to another behavior may be done inside a \blocdo~:
\begin{lstlisting}
behavior shiftInstructionType(out u16 source)
  ...
end behavior

behavior shiftInstructionBehavior(out u16 rdValue, u16 source)
  ...
end behavior

behavior shiftInstruction() #SHIFT
  ...
  u16 rdValue;
  u16 source;
  shiftInstructionType(source)
  shiftInstructionBehavior(rdValue, source)
  ...
end behavior
\end{lstlisting}
Here, in behavior {\tt shiftInstruction}, 2 variables  {\tt rdValue} and {\tt source} are defined and 2 behavior nodes  {\tt shiftInstructionType} and {\tt shiftInstruction\-Behavior} are called.  {\tt shiftInstructionType} gives a value to \texttt{source} that is then given to the other behavior  {\tt shiftInstructionBehavior} that gives a value to  {\tt rdValue}.

\subsection{Select structure}
\label{sec:behSelect}

The \texttt{select} structure allows to choose between different behaviors, based on the instruction tags:
\begin{lstlisting}
  select
    case #ROL do rdValue := ALU.ROL(rdValue, source); end do
    case #ROR do rdValue := ALU.ROR(rdValue, source); end do
  end select
\end{lstlisting}
Here, tags \texttt{ \#ROL} or \texttt{\#ROR} are used to distinguish the behavior of two instructions, and the \blocsdo\ differs in each part.

\begin{lstlisting}
  select
    case logicImmAndArithImmUpdateNoReg(rdValue, imm8)
    case logicImmAndArithImmUpdateReg(rdValue, imm8, rdIndex)
  end select;
\end{lstlisting}
Here, one or the other behavior is taken into account (each behavior will have references to \texttt{tags} that differs). Both types of \texttt{case} may be used in the same \texttt{select} structure.

A \texttt{select} structure should not appear in a \blocdo.

\subsection{\Blocdo}
\label{sec:behDo}
\blocsdo\ allow to write the instruction's algorithm. They may:
\begin{itemize}
\item access to components through their methods;
\item declare local variables;
\item perform assignments;
\item use control structures  {\em if ... then ... else}
\item use expressions as defined in section \ref{sec:expressions};
\end{itemize}
The following behavior is extracted from the XGate description:
\begin{lstlisting}
behavior loadStoreType(u1 accessType, u16 addr, u3 regIndex)
  select
    case #LOAD
      do
         if accessType = 0 then
           u8 val := mem.read8(addr);
           GPR.write8(regIndex, val);
         else
           u16 val := mem.read16(addr);
           GPR.write16(regIndex, val);
         end if;
      end do
    case #STORE 
      do
         if accessType = 0 then
           u8 val := GPR.read8(regIndex);
           mem.write8(addr, val);
         else
           u16 val := GPR.read16(regIndex);
           mem.write16(addr, val);
         end if;
      end do
  end select
end behavior
\end{lstlisting}
Here, one behavior is selected, based on the tag of the instruction (either \#LOAD or \#STO\-RE).  {\tt accessType} give the size of the memory access (8 or 16 bits) and methods related to component \texttt{GPR} are therefore called.

\section{Hardware components}
\label{sec:component}
An hardware component models an hardware part of the processor (ALU, memory, \ldots). its description is encapsulated and contains:
\begin{itemize}
\item member variables;
\item methods.
\end{itemize}
Here is an example with a component  \texttt{Fetcher} that manage the program counter:
\begin{lstlisting}
component Fetcher {
  program counter u16 pc; -- generate get and set methods.

  void reset() {
    pc := 0;
  }

  void branch(s16 offset) {
    pc := (u16)((s16)(pc) + offset);
  }
}
\end{lstlisting}
This component uses one member variable, which is a register (here a specific register; the program counter). There are 2 methods associated to this component \texttt{reset} and \texttt{branch}. syntax is very similar to C language. Components methods can be called inside the behavior description, inside a  \blocdo\ : \texttt{<componentName>.<methodName>}, for example: \texttt{Fetcher.branch(10)}.


\emph{This approach is very similar to an object approach, however, there is no concept instance associated with a component.} If there are 2 ALUs in a processor (superscalar architecture..), only one component is defined. This choice is done because instructions do not care about which ALU is used, it only have to perform an ALU related operation. The choice to use one ALU or the other depends on the micro-architecture implementation, and related constraints are defined in the micro-architecture part (not yet implemented!).

\subsection{Writing a method}

A method is defined as:
\begin{verbatim}
<returnValues> methodName(<inputParameters>)
{
  <implementation>
}
\end{verbatim}
where:
\begin{itemize}
\item \texttt{<returnValues>} is a comma separated lists of return types (\texttt{u8}, \texttt{s5},\ldots). If the method does not return anything, the \texttt{void} keyword is used;
\item \texttt{<inputParameters>} is a comma separated list of input parameters, with both their types and names
\item \texttt{<implementation>} is the same as in section \ref{sec:behDo}. However the \texttt{return} statement should return as many values as in the method prototype.
\end{itemize}

Basic example:
\begin{lstlisting}
  -- returns both the result and the carry flag
  u32,u1 addWithCarry(u32 ra, u32 rb) {
    u33 result := ra + rb
    return result{31..0},result{32}
  }
\end{lstlisting}

\subsection{Member variables}
Member variables are accessible only in methods of the same component (encapsulation). However, registers (see Section \ref{sec:defReg}) have a wider scope and can be used in any other place in the description.


\subsection{Using components}
Components fill the gap between the \emph{behavior description} and the \emph{micro-architecture description}. This last description that is under construction allows to describe the pipeline and other parts that can affect timings of the simulation.

When modeling an instruction, some of the behavior can be described either in a \texttt{behavior} or in a \texttt{component}. For instance, the implementation of a branch instruction can use either \texttt{Fetcher.branch(..)}, or modify directly the program counter inside the \texttt{behavior}. However, using a component allows to use the branch instruction inside the micro-architecture (the component method will be associated to a specific pipeline stage). Components methods are also used in the \texttt{timing} view for simple architectures (without any pipeline), see chapter \ref{chap:timing}.

This approach is also useful for behaviors that have not temporal impact on the simulation: a post-incremented instruction with a dedicated hardware for instance. In that last case, the addition can be declared inside the behavior, as it will never affect timings.

%\section{Exemple de mise en œuvre}


%!TEX root = ./main.tex
%!TEX encoding = UTF-8 Unicode
\chapter{Description of instructions syntax code}
\label{chap:syntax}
\section{Introduction}
In this chapter, we will present how to describe the instruction set syntax for a given processor. To illustrate, we will use different examples extracted from the description of the AVR micro-controller, the Freescale HCS12 processor and its co-processor XGate. 

The syntax view follows the same principle as the other two views (format and behavior). It describes the textual format of instructions by concatenating strings. In other words, it assigns a textual syntax for each instruction signature to allow, for example, the disassembling.

\section {Arborescent structure of the syntax view in \harmless}
As in the other two views (format and behavior), the syntax view consists of a set of trees, where a node describes a part of the syntax of one or more instructions. Each branch of the tree represents an instruction.

In this view, each node is written using the following syntax:

\begin{lstlisting}
syntax <name> [tag]
  <syntaxBody>
end syntax
\end{lstlisting}

A syntax node can call another node, or make a selection structure between different nodes, using the \texttt{select} structure type. This approach is common to the 4 views (\texttt{format}, \texttt{syntax}, \texttt{behavior} and \texttt{timing}) and is described in more details in section \ref{sec:modelisationArborescente}.

%Nous allons ici nous intéresser plus particulièrement au corps d'un nœud \texttt{<syntaxBody>} qui est spécifique à la génération d'un mnémonique, tout en donnant quelques exemples liés à la structure arborescente de la description des éléments de type \texttt{syntax}.
%
%\section{La partie \texttt{<syntaxBody>}}
%
%La partie \texttt{<syntaxBody>} est une succession d'éléments de type (\ref{sec:syntaxBody}):
%\begin{itemize}
%\item \emph{étiquette};
%\item étiquette (l'ensemble des étiquettes définissant la signature de l'instruction), voir \ref{sec:syntaxEtiquette};
%\item appel à un autre nœud de type \texttt{syntax}, en utilisant le nom de l'élément \texttt{syntax} appelé;
%\item structure de sélection, en utilisant le mot clé \texttt{select}, voir section \ref{sec:syntaxSelect};
%\item déclaration de \emph{field}, voir section \ref{sec:syntaxField};
%\item chaîne de caractères suivie ou non d'un ou plusieurs variables déclarées. Cette chaîne constitue le mnémonique à proprement parler de l'instruction, voir section \ref{sec:syntaxChaineCaract};
%\item structure de sélection permettant de mettre en œuvre un choix dynamiquement (en fonction de la valuer des données), voir section \ref{sec:syntaxIf}.
%\end{itemize}
%
%\subsection{Les étiquettes}
%\label{sec:syntaxEtiquette}
%Comme dans la vue binaire, les nœuds sont associés à des étiquettes (tags) dont l'ensemble forme la signature de l'instruction. Lors de l'évaluation des différents chemins des instructions, les étiquettes sont récoltées pour former la signature de l'instruction. On s'intéresse à un \emph{ensemble}, donc le nombre d'étiquettes de même nom n'est compté qu'une fois, et il n'y a pas de relation d'ordre, voir \ref{sec:signature}.
%
%\subsection{Structure de sélection \texttt{select}}
%\label{sec:syntaxSelect}
%Le mot clé {\tt "select"} permet de créer différentes branches dans l'arborescence, comme indiqué dans \ref{sec:modelisationArborescente}.
%
%La syntaxe générale de cette structure est la suivante:
%\begin{lstlisting}
%syntax test
%  <syntaxBody1>
%  select 
%    case <syntaxBody2>
%    case <syntaxBody3>
%    ...
%  end select;
%  <syntaxBody4>
%end syntax
%\end{lstlisting}
%Ainsi, dans cette architecture, 2 instructions sont définies. Elles sont composées respectivement des éléments:
%\begin{itemize}
%\item \texttt{<syntaxBody1>}, \texttt{<syntaxBody2>} et \texttt{<syntaxBody4>};
%\item \texttt{<syntaxBody1>}, \texttt{<syntaxBody3>} et \texttt{<syntaxBody4>};
%\end{itemize}
% Ainsi, la structure \texttt{select} permet de mutualiser les parties communes aux mnémoniques de différentes instructions (\texttt{<syntaxBody1>} et \texttt{<syntaxBody4>}), et de les différencier sur d'autres parties.
%
%Bien entendu, les étiquettes dans les parties \texttt{<syntaxBody2>} et  \texttt{<syntaxBody3>} doivent être différentes pour différencier les différents chemins (et constituer les signatures des instructions).
%
%Soit par exemple (issu du jeu d'instructions de l'AVR):
%\begin{lstlisting}
%syntax classicImm8 
%  field u4 regIndex -- should add 16 to the result
%  field u8 k
%  select
%    case #ANDI "ANDI" -- or CBR with ~k 
%    case #CPI  "CPI"
%    case #ORI  "ORI"
%    case #SBCI "SBCI"
%    case #SUBI "SUBI"
%  end select
%  " R\d, \x", regIndex+16, k
%end syntax
%\end{lstlisting}
%Cet exemple permet de construire la syntaxe de 5 instructions différentes, en mutualisant la récupération des champs issus du \texttt{format}, ainsi que leur affichage. Dans ce cas, les signatures des instructions ne comprennent qu'une seule étiquette.
%
%Cette structure en \texttt{select} est résolue de manière statique à la génération du simulateur. Par conséquent, 5 instructions seront générées, avec chacune leur code associé. Aucun impact sur les performances n'est associé à la multiplication des éléments \texttt{select}.
%
%\subsection{Récupération des champs du format binaire}
% \label{sec:syntaxField}
% Dans la vue syntaxique, il est nécessaire de pouvoir récupérer la valeur des différents champs extrait du code binaire de l'instruction. L'opération d'extraction de ces champs dans la partie binaire de l'instruction est réalisé dans la partie \texttt{format} de la description. Dans la partie \texttt{syntax}, on se contente de récupérer ces différents champs sous la forme de constantes.
% 
% Ainsi, le mot clé {\tt "field"} permet de référencer un champ qui est extrait de l'instruction dans la vue binaire.\harmless\ étant un langage fortement typé, le type des différents champs doit correspondre exactement avec celui extrait de la vue binaire:
%
%\begin{lstlisting}
%field <typeDonnee> <nom_du_champ>
%\end{lstlisting}
%
%Où {\tt <typeDonnee>} est un type de donnée, comme défini dans la section \ref{sec:TypeDonnees} (\texttt{u3} pour un entier non signé sur 3 bits, \texttt{s5} pour un entier signé sur 5 bits, \ldots). Il doit correspondre au nombre de bits extraits dans la vue binaire pour la même variable (ou être plus grande).
%
%Par exemple, une donnée non signée nommée {\tt "rs1Index"} est déclarée dans la vue binaire de la façon suivante:
%
%\begin{lstlisting}
%rs1Index := slice{7..5}
%\end{lstlisting}
%
%Dans la vue syntaxique, sa valeur sera importée sous la forme:
%
%\begin{lstlisting}
%field u3 rs1Index
%\end{lstlisting}
%
%Ici, nous pouvons remarquer que la taille du champ est égale à 3, ce qui correspond bien au 3 bits extraits (bit 7, 6 et 5).
%
%\subsection{Les  chaînes de caractères}
%\label{sec:syntaxChaineCaract}
%Dans la vue syntaxique, une chaîne de caractères est une suite de caractères entre guillemets. 
%
%Lors de l'évaluation d'une branche de la description (qui modélise une instruction), les différentes chaînes de caractères sont concaténées pour former la mnémonique de l'instruction.
%
%Avec une approche similaire à celle de la fonction \texttt{printf} du langage \texttt{C}, il est possible de faire référence à des données qui sont définies en dehors de la chaîne de caractères, en utilisant des séquences d'échappement. Chaque séquence d'échappement est alors remplacée par une expression:
%
%\begin{lstlisting}
%<chaine_de_caracteres>, <expression_0>, ..., <expression_n>
%\end{lstlisting}
%
%soir par exemple:
%\begin{lstlisting}
%    "CLR R\d", rdIndex
%    ...
%    "LDI R\d, \x", regIndex+16, k
%\end{lstlisting}
%Il est indispensable d'avoir le même nombre de caractères d'échappements que d'expressions.
%
%\subsubsection{Séquences d'échappement dans les chaînes de caractères}
%Les séquences d'échappement permettent de faire référence aux expressions qui suivent la chaîne de caractère. Les 4 séquences d'échappement permettent de définir la base utilisée pour l'affichage:
%\begin{table}[!h]
%\begin{center}
%\begin{tabularx}{0.7 \columnwidth}{|X|c|}
%\hline
%\bf séquence d'échappement & \bf Signification \\  \hline
%\textbackslash b & Base binaire \\ \hline
%\textbackslash d & Base décimal \\ \hline
%\textbackslash x & Base hexadécimal \\ \hline
%\textbackslash o & Base octal \\ \hline
%\end{tabularx}
%\caption{Base utilisée pour les différentes séquences d'échappement}
%\label{tab:type-donnee}
%\end{center}
%\end{table}
%
%De plus, pour chaque base, \harmless\ donne la possibilité de préciser les caractères avant (\texttt{prefix}) et après (\texttt{suffix}) l'expression de manière globale, selon la syntaxe suivante: 
%
%\begin{lstlisting}
%number syntax <base> <prefix_ou_suffix> <chaine_de_caracteres>
%\end{lstlisting}
%
%Où {\tt <base>} peut être {\tt binary, octal, hexadecimal} ou {\tt decimal}. {\tt <prefix\_ou\_suffix>} peut être {\tt prefix} ou {\tt suffix}, et \texttt{<chaine\_de\_caracteres>} est une suite quelconque de caractères entres guillemets. Ces directives sont généralement placées au début de la section où sont placés les éléments \texttt{syntax}.
%Voici un exemple: 
%
%\begin{lstlisting}
%number syntax octal prefix "o"
%number syntax hexadecimal prefix "0x"
%number syntax hexadecimal suffix "h"
%\end{lstlisting}
%
%Ainsi, à travers cet exemple, la description suivante:
%\begin{lstlisting}
%    "LDI R\d, \x", regIndex+16, k
%\end{lstlisting}
%conduira à l'affichage suivant (si k=5, et regIndex = 3):
%\begin{verbatim}
%    "LDI R19, 0x5h"
%\end{verbatim}
%
%\subsection{La structure \texttt{if ... then ... else}}
%\label{sec:syntaxIf}
%La structure conditionnelle \texttt{if ... then ... else} permet de faire un choix dynamique en fonction des différents champs d'une instruction. Elle est utilisée de la manière suivante:
%\begin{lstlisting}
%if <condition> then
%  <ifSyntaxStatement>
%[elseif <condition> then
%  <ifSyntaxStatement>
%]
%[else 
%  <ifSyntaxStatement>
%]
%end if;
%\end{lstlisting}
%
%La \texttt{<condition>} est une expression booléenne classique (voir \ref{sec:expressions}).
%La partie \texttt{<ifSyntaxStatement>} n'est prise en compte que si la condition est vraie. On peut avoir:
%\begin{itemize}
%\item une chaîne de caractères;
%\item une autre condition (\texttt{if..then..end if}.
%\end{itemize}
%
%Par contre, comme la condition est évaluée dynamiquement (à l'exécution), il n'est pas possible de faire intervenir des informations qui sont utilisées statiquement (au moment de la génération des sources du simulateur), comme par exemple une étiquette ou une structure de type \texttt{select}.
%
%Les parties {\tt "elseif"} et {\tt "else"} sont facultatives.
%
%Cette approche permet notamment de définir les \emph{mnémoniques simplifiées} associées à certaines instructions. Dans l'exemple suivant, une instruction de type \texttt{OR Rd, Ra, Ra} est simplifiée en \texttt{MOV Rd, Ra}:
%\begin{lstlisting}
%syntax orOperation #TriadicInst #OR
%  field u3 rs1Index
%  field u3 rs2Index
%  field u3 rdIndex
%  if rs1Index = rs2Index then
%    "MOV R\d, R\d", rdIndex, rs1Index
%  else
%    "OR R\d, R\d, R\d", rdIndex, rs1Index, rs2Index
%  end if;
%end syntax
%\end{lstlisting}
%
%\section{Assemblage du tout}
%
%Prenons comme exemple l'ensemble d'instructions faisant intervenir 3 registres dans la XGate. Dans la documentation technique, cet ensemble d'instructions est résumé par le tableau \ref{tab:triadic}.
%
%\begin{table}[!h]
%\begin{center}
%\begin{tabularx}{1.1 \columnwidth}{|c|c|c|c|c|c|c|c|c|c|c|c|c|c|c|c|X|} 
%\hline
%\bf Functionality & \bf 15 & \bf 14 & \bf 13 & \bf 12 & \bf 11 & \bf 10 & \bf 9 & \bf 8 & \bf 7 & \bf 6 & \bf 5 & \bf 4 & \bf 3 & \bf 2 & \bf 1 & \bf 0\\  \hline
%\bf Logical triadic & \multicolumn{16}{c|}{ } \\ \hline
%AND RD, RS1, RS2 & 0 & 0 & 0 & 1 & 0 & \multicolumn{3}{c|}{RD} & \multicolumn{3}{c|}{RS1} & \multicolumn{3}{c|}{RS2} & 0 & 0 \\ \hline
%OR RD, RS1, RS2 & 0 & 0 & 0 & 1 & 0 & \multicolumn{3}{c|}{RD} &  \multicolumn{3}{c|}{RS1} & \multicolumn{3}{c|}{RS2} & 1 & 0 \\ \hline
%XNOR RD, RS1, RS2 & 0 & 0 & 0 & 1 & 0 & \multicolumn{3}{c|}{RD} & \multicolumn{3}{c|}{RS1} & \multicolumn{3}{c|}{RS2} & 1 & 1 \\ \hline
%\bf Arithmetic triadic & \multicolumn{16}{c|}{For compare use SUB R0,Rs1,Rs2} \\ \hline
%SUB RD, RS1, RS2 & 0 & 0 & 0 & 1 & 1 & \multicolumn{3}{c|}{RD} & \multicolumn{3}{c|}{RS1} & \multicolumn{3}{c|}{RS2} & 0 & 0 \\ \hline
%SBC RD, RS1, RS2 & 0 & 0 & 0 & 1 & 1 & \multicolumn{3}{c|}{RD} & \multicolumn{3}{c|}{RS1} & \multicolumn{3}{c|}{RS2} & 0 & 1 \\ \hline
%ADD RD, RS1, RS2 & 0 & 0 & 0 & 1 & 1 & \multicolumn{3}{c|}{RD} & \multicolumn{3}{c|}{RS1} & \multicolumn{3}{c|}{RS2} & 1 & 0 \\ \hline
%ADC RD, RS1, RS2 & 0 & 0 & 0 & 1 & 1 & \multicolumn{3}{c|}{RD} & \multicolumn{3}{c|}{RS1} & \multicolumn{3}{c|}{RS2} & 1 & 1 \\ \hline
%\end{tabularx}
%\caption{Ce tableau représente l'ensemble d'instructions  {\tt "triadic"}, comme trouvé dans le datasheet de la XGate.}
%\label{tab:triadic}
%\end{center}
%\end{table}
%
%Dans  \harmless\ , cet ensemble d'instructions peut être décrit comme suit:
% 
%\begin{lstlisting}
%syntax triadicInstruction #TriadicInst 
%  field u3 rs1Index
%  field u3 rs2Index
%  field u3 rdIndex
%  triadicOperation
%  " R\d, R\d, R\d", rdIndex, rs1Index, rs2Index
%end syntax
%
%syntax triadicOperation 
%  select
%    case #AND  "AND"
%    case #XNOR "XNOR"
%    case #SBC  "SBC"
%    case #ADD  "ADD"
%    case #ADC  "ADC"
%  end select
%end syntax
%
%syntax orOperation #TriadicInst #OR
%  field u3 rs1Index
%  field u3 rs2Index
%  field u3 rdIndex
%  if rs1Index = rs2Index then
%    "MOV R\d, R\d", rdIndex, rs1Index
%  else
%    "OR R\d, R\d, R\d", rdIndex, rs1Index, rs2Index
%  end if;
%end syntax
%
%syntax subOperation #TriadicInst #SUB
%  field u3 rs1Index
%  field u3 rs2Index
%  field u3 rdIndex
%  if rdIndex = 0 then
%    "CMP R\d, R\d",rs1Index, rs2Index
%  else
%    "SUB R\d, R\d, R\d", rdIndex, rs1Index, rs2Index
%  end if;
%end syntax
%
%\end{lstlisting}
%
%La vue syntaxique a une structure arborescente, dont la représentation graphique est donnée dans le schéma \ref{fig:TriadicInst}. On peut noter que les 2 premiers éléments \texttt{syntax} permettent de traiter les cas généraux, et que les 2 suivants permettent de traiter des cas particuliers (car contenant des mnémoniques simplifiées).
%
%\begin{figure}		
%  \begin{center}
%    \includegraphics[width=0.5 \linewidth]{../common/images/TriadicInst.pdf}
%    \caption{Syntaxe de l'ensemble d'instructions triadic.}
%    \label{fig:TriadicInst}
%  \end{center}
%\end{figure}

%!TEX root = ./main.tex
%!TEX encoding = UTF-8 Unicode
\chapter{Description fonctionnelle des éléments de mémorisation.}
\label{sec:mem_program}

La description des éléments de mémorisation est nécessaire pour la génération d'un simulateur de jeu d'instruction, mais aussi d'un simulateur précis au cycle. Dans cette dernière approche, il sera de plus nécessaire de rajouter des informations relatives au comportement temporel de la mémoire.

Dans cette version, la hiérarchie mémoire n'est pas modélisée (système de cache). En effet, son comportement est normalement transparent d'un point de vue fonctionnel, bien qu'il y ait la possibilité de gérer quelquefois les caches (cache locking, scratch pad, \ldots).

\section{Généralités}
\subsection{Organisation}
La mémoire est définie à l'intérieur d'un composant (voir section \ref{sec:component}). Ceci présente un double avantage:
\begin{itemize}
\item il est possible de rajouter des méthodes dans le composant, afin de simplifier l'accès à la mémoire (rajout de méthodes \texttt{push/pop} pour modéliser une pile, modélisation de mémoire paginée, \ldots
\item il est aussi possible de permettre l'accès à plusieurs zones de mémoire de manière transparente. Par exemple, une zone de mémoire RAM (largeur 16 bits) et une EEPROM (largeur 8 bits) dans le même espace d'adressage.
\end{itemize}

Une zone de mémoire est déclarée en utilisant le mot clé \texttt{memory}. Certains paramètres permettent de définir la zone mémoire. 

\subsubsection{Plage d'adresse}
La plage d'adresse est définie de la manière suivante: \texttt{address := startAddr..endAddr}, où \texttt{startAddr} et \texttt{endAddr} sont des valeurs numériques. La zone de mémoire est alors définie sur la zone d'adresse qui commence à \texttt{startAddr} \emph{inclus}, et va jusqu'à \texttt{endAddr} \emph{exclus}. Il est à noter qu'il est possible d'utiliser les suffixes \texttt{kb} et \texttt{mb} au valeurs numériques (voir section
\ref{sec:nombres}):
\begin{lstlisting}
    address := 0..128kb  
\end{lstlisting}

\subsubsection{Largeur du bus mémoire}
La largeur d'accès est définie par le mot clé \texttt{width}. Elle permet de définir la largeur maximale d'un accès en mémoire. L'accès reste possible pour les puissances de 2 inférieures. Par exemple:
\begin{lstlisting}
    width := 20  
\end{lstlisting}
L'accès sera possible sur 20 bits, mais aussi sur 16 bits et 8 bits.

\subsubsection{Type de mémoire}
Le type de mémoire est actuellement \texttt{RAM}, \texttt{ROM}, ou \texttt{register}. Dans le cas de la ROM, il n'est pas possible de faire des accès en écriture, mais il reste possible de mettre le code programme au démarrage, voir la section \ref{sec:memProgramme}. Exemple:
\begin{lstlisting}
    type := RAM
\end{lstlisting}

Il n'y a actuellement pas de différence entre les types \texttt{RAM} et \texttt{register}.

\subsubsection{Décalage}
Il est possible de faire un décalage des adresses pour une utilisation simplifiée, à l'aide du mot clé \texttt{stride}. Soit par exemple une zone de 16 registres généraux sur un processeur 16 bits. Ils peuvent être défini de la manière suivante:
\begin{lstlisting}
    memory GPR {
      width   := 16    -- get 16 bits / access
      address := 0..31 -- 32 octets : 16 registres de 16 bits
      stride  := 2 
      type    := register 
    }
\end{lstlisting}
Ainsi dans cet exemple, l'accès en '5' permettra une lecture dans la mémoire à l'adresse 10 qui contient réellement le contenu du registre GPR 5.
Le stride doit forcément être une puissance de 2. Dans le cas contraire, une erreur est générée.

\subsection{Sous blocs mémoire}
À l'intérieur d'une zone mémoire, il est possible de définir un sous bloc qui va avoir des caractéristiques différentes. Par défaut, un sous-bloc \emph{hérite} des paramètres du bloc dans lequel il est inséré. 

D'autre part, et il est possible de mapper le sous-bloc à l'intérieur du bloc principal.
Par exemple:
\begin{lstlisting}
  memory ram {
    width   := 16  -- get 16 bits / access
    address := \x0..\x10FF 
    type    := RAM 
   
    sfr { -- can be accessed by IN/OUT instructions.
      address := 0..\x3F 
      type    := register 
    } maps to \x20 
  }
\end{lstlisting}
Cet exemple tiré de la description de l'AVR met en évidence:
\begin{itemize}
\item la plage mémoire de \texttt{0} à \texttt{0x10FF} est définie comme de la RAM,
\item la plage mémoire de \texttt{0x20} à \texttt{0x20+0x3F} est \emph{redéfinie} comme étant de type \texttt{register}.
\item un accès à l'adresse \texttt{0x30} de la zone \emph{ram} correspond à la même place qu'un accès à l'adresse \texttt{0x10} (à cause de l'offset de \texttt{0x20}) d'un \emph{sfr}.
\end{itemize}
Ce dernier point est le plus important, car lors de la description du jeu d'instruction, il ne sera pas nécessaire de faire de décalage dans la \emph{ram} lorsqu'on fera un accès à un registre de type \emph{sfr}.

D'une manière plus générale, un sous-bloc est défini de la manière suivante:
 \begin{lstlisting}
    sousBloc { 
      <...>
    } maps to <expression> 
  }
\end{lstlisting}
Ceci permet notamment l'utilisation de sous-blocs dont la position n'est pas constante. Par exemple, dans l'infineon C166, les registres généraux sont définis par rapport au registre \texttt{CP} (pour \texttt{Context Pointer}), on peut alors définir une tel zone de la manière suivante:
 \begin{lstlisting}
   memory internalRam {
    address    := \x1000..\x3FFF;
    type    := RAM;
    width   := 16;

    GPR {   -- definition relative a CP (Context Pointer)
      type    := register;
      address := 0..31; -- 16 registres
      stride  := 1
    } maps to CP
  }
}
\end{lstlisting}
On considère ici que le registre CP est déclaré, voir section \ref{sec:defReg}
Dans ce dernier cas, un accès au GPR 3 sera identique à un accès à la mémoire CP + 3*2 (car il y a un stride de 1).

\emph{Note:} En l'absence du \texttt{maps to <expression>}, le sous-bloc est mappé à l'adresse 0.
\subsection{Mémoire programme}
\label{sec:memProgramme}
Plusieurs zones mémoires sont définies, et celle-ci peuvent avoir des plages d'utilisation qui se recouvrent. C'est par exemple le cas sur une architecture Harvard où les mémoire d'instruction et de données sont séparées. Un banc de registres est aussi vu comme une zone de mémoire (qui commence généralement à l'adresse 0).
Ainsi, toutes les zones mémoire qui peuvent être initialisée avec du code programme sont précédée du mot clé \texttt{program}. Il ne doit pas y avoir de recouvrement de ces zones:

\begin{lstlisting}
program memory flash {
  width   := 16  -- get 16 bits / access
  address := 0..128kb  -- 128k
  type    := ROM
}
memory ram {
  width   := 16  -- get 16 bits / access
  address := \x0..\x10FF 
  type    := RAM 
}  
\end{lstlisting}
Dans cet exemple, le programme pourra être mis dans la mémoire \emph{flash}. Ceci lève l'ambiguïté car la \emph{ram} est mappée sur la même zone mémoire (architecture harvard).

\subsection{Accès dans les composants}
Lors de la définition d'une zone mémoire, un certain nombre de fonctions d'accès sont générées automatiquement. Soit par exemple l'architecture typique suivante, avec 2 zones mémoire \texttt{mem} et \texttt{mem2} qui sont définies dans un composant (c'est obligatoire), avec \texttt{mem} qui comporte un sous bloc \texttt{subMem}:
\begin{lstlisting}
component comp
  memory mem {
    width   := 16  -- get 16 bits / access
    address := 0..128kb  
    type    := RAM
    
    subMem {
      address := 0..32
      type := register
    } -- pas de 'maps to', donc par defaut mapping en 0.
  }
  memory mem2 {
    width   := 16  -- get 16 bits / access
    address := 256kb..512kb  
    type    := flash
  }
}
\end{lstlisting}
Il ne doit pas y avoir de recouvrement entre les zones mémoire qui sont définies dans un composant.

Comme largeur de bus est de 16 bits, les accès pourront se faire sur 8 ou 16 bits (soit la taille spécifiée, ainsi que les puissances de 2 inférieures). Les méthodes d'accès générées automatiquement sont alors en lecture:
\begin{itemize}
\item \texttt{u16 comp.read16(u32 address)} Renvoie la valeur d'un élément dans le composant, sur 16 bits. Suivant l'adresse, une valeur de la zone mémoire \texttt{mem} ou \texttt{mem2} sera renvoyée. Si l'adresse est invalide (ne correspond pas à une zone mémoire), la valeur 0 est renvoyée;
\item \texttt{u8  comp.read8(u32 address)} Renvoie la valeur d'un élément dans le composant, sur 8 bits, de la même manière que la méthode précédente;
\item \texttt{u16  comp.mem.read16(u32 address)} Renvoie la valeur d'un élément dans l'élément mémoire \texttt{mem} du composant \texttt{comp}, sur 16 bits;
\item \texttt{u8  comp.mem.read8(u32 address)} Renvoie la valeur d'un élément dans l'élément mémoire \texttt{mem} du composant \texttt{comp}, sur 8 bits;
\item \texttt{u16  comp.mem.subMem.read16(u32 address)} Renvoie la valeur d'un élément dans le sous élément mémoire \texttt{subMem} du composant \texttt{comp}, sur 16 bits;
\item \texttt{u8  comp.mem.subMem.read8(u32 address)} Renvoie la valeur d'un élément dans le sous élément mémoire \texttt{subMem} du composant \texttt{comp}, sur 8 bits;
\end{itemize}
Les fonctions d'écritures sont fournies sur le même principe:
\begin{itemize}
\item \texttt{void  comp.write8(u32 address, u8 val)} Écrit la valeur \texttt{val} à l'adresse \texttt{address} de l'élément mémoire \texttt{mem} ou \texttt{mem2} suivant l'adresse. En cas d'adresse invalide, une erreur sera générée (si les options de compilations sont correctement positionnées, voir \ref{sec:cflags});
\item \texttt{void  comp.write16(u32 address, u16 val)} idem sur 16 bits;
\item \texttt{void  comp.mem.write8(u32 address, u8 val)} suivant le même principe que pour la lecture;
\item \texttt{void  comp.mem.write16(u32 address, u16 val)} suivant le même principe que pour la lecture;
\item \texttt{void  comp.mem.subMem.write8(u32 address, u8 val)} suivant le même principe que pour la lecture;
\item \texttt{void  comp.mem.subMem.write16(u32 address, u16 val)} suivant le même principe que pour la lecture;
\end{itemize}

\section{les registres}
\label{sec:defReg}
Les registres sont utilisées à de nombreux endroits dans la description, que ce soit dans les composants, la mémoire, et même la vue comportementale de la description.

C'est pourquoi les registres sont dans une certaine mesure une entorse à l'encapsulation qui est présentée dans les composants: 

\emph{Un registre, qu'il soit définit dans un composant ou dans une zone mémoire est accessible de manière globale dans toute la description.}

\subsection{Définition dans un composant}
Dans un composant, il est déclaré en utilisant le mot clé \texttt{register}:
\begin{lstlisting}
register u8 SP
\end{lstlisting}
Cet exemple défini un registre nommé SP (de 8 bits) qui est accessible dans tous les composants, en utilisant directement son nom, dans une zone d'implémentation:
\begin{lstlisting}
  SP := SP+1
\end{lstlisting}
On peut définir un registre en nommant des champs de bits:
\begin{lstlisting}
  register u16 T01CON {
    T0I := slice{2..0}  -- 3 bits
    T0M := slice{3}     -- 1 bit
    T0R := slice{6}     -- 1 bit
    T1I := slice{10..8} -- 3 bits
    T1M := slice{11}    -- 1 bit
    T1R := slice{14}    -- 1 bit
  }
\end{lstlisting}
L'accès à un champ de bit se fait alors de la manière suivante:
\begin{lstlisting}
  T01CON.T0R := 1;
\end{lstlisting}

\subsection{Définition dans un bloc mémoire}
\subsubsection{Cas général}
Un registre défini dans une zone mémoire est mappé en mémoire, il est donc nécessaire de préciser en quelle adresse il est mappé:
\begin{lstlisting}
component sram {
  memory ram {
    width   := 16  -- get 16 bits / access
    address := \x0..\x10FF 
    type    := RAM 

    register u8  SPH  maps to \x5e  -- stack (high byte)
    register u8  SPL  maps to \x5d  -- stack (low byte)
    register u16 SP   maps to \x5d  -- stack (16 bits)
    ...
  }
}
\end{lstlisting}
Si la taille du registre n'est pas spécifiée, c'est la largeur de bus qui est utilisée (en non signé).

Ainsi, dans les 2 lignes suivantes... sont identiques:
\begin{lstlisting}
  SP := SP+1
  sram.ram.write16(\x5d, sram.ram.read16(\x5d)+1)
\end{lstlisting}

\subsubsection{Accès à un champ de bits}
Comme pour les registres définis dans un composant, il est possible de définir des accès à des champs de bit:
\begin{lstlisting}
    register u8  CCR  maps to \x5f {
      C := slice{0} -- carry flag
      Z := slice{1} -- zero flag
      N := slice{2} -- neg flag
      V := slice{3} -- overflow flag
      S := slice{4} -- sign bit
      H := slice{5} -- half carry flag
      T := slice{6} -- Bit copy storage
      I := slice{7} -- global interrupt flag
    } 
\end{lstlisting}

\subsubsection{Registre constant}
Il est courant d'avoir un processeur qui contient certains registres constant (cas du registre \texttt{PVR} du PowerPC par exemple).
\begin{lstlisting}
    register u32 PVR maps to 0 is read \x00800200;
\end{lstlisting}

\subsection{Le compteur programme}
Le compteur programme est un registre un peu spécifique dans \harmless. En effet, lors du décodage, le compteur programme est incrémenté de manière implicite dans le décodeur. Ainsi, il est nécessaire pour \harmless\ de connaître ce registre. Il est déclaré en utilisant \texttt{program counter} à la place de \texttt{register}:
\begin{lstlisting}
  program counter u32 PC
\end{lstlisting}

Attention, il \emph{doit} y avoir \emph{un et seulement un} compteur programme déclaré dans la description (si on compile la partie \emph{behavior}).

\subsection{Initialisation}
L'initialisation des registres peut être réalisée:
\begin{itemize}
\item dans la fonction d'initialisation des composants (voir section \ref{sec:initComponent}), pour les registres qui sont définis dans les composants
\item directement pour les registres qui sont mappés en mémoire (bien qu'il soit aussi possible de les initialiser dans les composants.
\end{itemize}
Pour les registres qui sont mappé en mémoire, il est possible de les initialiser avec la syntaxe suivante (dans un élement \texttt{memory}):
\begin{lstlisting}
    register u16 X maps to \x1a := \xa5a5
\end{lstlisting}

Un registre non initialisé explicitement est fixé à 0. Toutefois, cette valeur peut être mise à jour dans une méthode \texttt{reset} dans un des composants.

\section{Exemples}
\subsection{Accès à une zone de pile}
il est possible dans un composant de mettre des méthodes en plus d'une zone mémoire. Ceci peut notamment servir à enrichir l'accès à la mémoire. Soit l'exemple d'une zone de pile:
\begin{lstlisting}
component sram {
  memory ram {
    width   := 16  -- get 16 bits / access
    address := \x0..\x10FF 
    type    := RAM 

    register u8  SPH  maps to \x5e  -- stack (high byte)
    register u8  SPL  maps to \x5d  -- stack (low byte)
    register u16 SP   maps to \x5d  -- stack (16 bits)
    ...
  }
  void push(u8 val) {     -- post decrement
    sram.write8(SP, val) 
    SP := SP-1 
  }

  u8 pop() {
    u8 result 
    SP := (u16)(SP+1) 
    result := sram.read8(SP) 
    return result 
  }
}
\end{lstlisting}
Dans cet exemple, l'ajout de méthodes relatives à l'accès mémoire présente tout son intérêt.

\subsection{Mémoire paginée}
Dans le cas de la mémoire paginée, l'adresse réelle est connue en faisant une opération entre la valeur de l'adresse passée en paramètre, et la valeur d'un registre. Soit l'exemple partiel suivant inspiré de l'espace d'adressage du MC9S12XDP512 (HCS12X avec 512 kb de flash).
\begin{lstlisting}
component Mem {
  memory registers {
    width := 16;
    address := \x0..\x7FF; -- 2 kb of registers
    type := register;

    register u16 GPAGE maps to \x10; -- Global page index
    register u8  RPAGE maps to \x16; -- RAM page index
    register u8  EPAGE maps to \x17; -- EEPROM Page index
    register u8  PPAGE maps to \x30; -- Program page index
  }
  program memory windowedEeprom {
    width := 16;
    address := \x13_F000..\x13_FFFF; -- 4kb
    type := ROM;
  }
  program memory nonWindowedEeprom {
    width := 16;
    address := \x0C00..\x0FFF; -- 1kb
    type := ROM;
  }
  program memory windowedFlash {
    width := 16;
    address := \x78_0000..\x7F_FFFF; -- 512 kb of flash.
    type := ROM;
  }
  program memory secondUnpagedflash { --including interrupt vectors.
    width := 16;
    address := \xC000..\xFFFF; -- 16 kb of flash.
    type := ROM;
  }
  ...
  u8 memRead8(u16 addr)
  {
    -- description from MC9S12XDP512 Data Sheet, p.31
    u8 val;
    if     addr < \x0800 then val := Mem.registers.read8(addr);
    elseif addr < \x0c00 then val := Mem.windowedEeprom.read8((EPAGE cat addr{9..0}) | \x100000);
    elseif addr < \x1000 then val := Mem.nonWindowedEeprom.read8(addr)
    elseif addr < \xC000 then val := Mem.windowedFlash.read8((PPAGE{7..0} cat addr{13..0}) | \x400000);
    elseif addr < \xFFFF then val := Mem.secondUnpagedflash.read8(addr);
    end if;
    return val;
  }
  ...
}
\end{lstlisting}
Le même type de méthode que memRead8 doit être fait pour les accès sur 16 bits, ainsi que les accès en écritures.


%!TEX root = ./main.tex
%!TEX encoding = UTF-8 Unicode
\chapter{Vue temporelle basique}
\label{chap:timing}
La vue temporelle basique sert à modéliser le temps mis par le processeur pour l'exécution de chaque instruction. Cette approche sert uniquement dans le cas d'une architecture simple, sans pipeline, dont le temps d'exécution d'une instruction ne dépend pas des instructions précédentes.


\section{Architecture générale}
La vue \texttt{timing} vient se rajouter aux vues \texttt{format}, \texttt{behavior} et \texttt{syntax} qui permettent de décrire le jeux d'instruction. Tout comme ces vues, on retrouve une structure arborescente lors de la description des instructions.

La structure générale des nœuds de description des timing (comme les autres vues) est de la forme:
\begin{lstlisting}
timing <name> [etiquette]
  <timingBody>
end timing
\end{lstlisting}

la partie \texttt{<timingBody>} regroupe:
\begin{itemize}
\item \emph{étiquette};
\item appel à un autre nœud de type timing;
\item structure de sélection, en utilisant le mot clé \texttt{select}. Voir section \ref{sec:timingSelect};
\item partie d'implémentation des timings, voir section \ref{sec:timingDo};
\item partie relative à l'accès à un composant, voir section \ref{sec:timingMethodAccess}
\end{itemize}

\subsection{Structure de sélection}
\label{sec:timingSelect}
L'utilisation de la sélection (différentes branches de l'arbre) se fait à travers l'instruction \texttt{select}, comme pour chaque vues de la description:
\begin{lstlisting}
  select 
    case <timingBody>
    case .. 
  end select
\end{lstlisting}

Un changement \emph{majeur} toutefois: Une instruction est représentée par un ensemble d'étiquette (i.e. la signature de l'instruction). Pour repérer une instruction dans une des 3 vues (\texttt{format}, \texttt{behavior} et \texttt{syntax}), il faut que la signature de l'instruction soit la même. l'approche est un peu différente dans la vue \texttt{timing}, car on peut utiliser uniquement un sous ensemble de la signature de l'instruction. Si plusieurs chemins conviennent, c'est celui qui a le plus d'étiquettes de l'instruction qui sera pris en compte.

Soit par exemple l'instruction \texttt{i1} ayant pour signature \texttt{\#A \#B \#C} et l'instruction \texttt{i2} ayant pour signature \texttt{\#A \#D}, associée à la description:

\begin{lstlisting}
timing t1
  #A
  select 
    case #D -- chemin 1
    case      -- chemin 2
  end select
end timing
\end{lstlisting}
À l'issue de cette description, 2 chemins sont possibles, avec les étiquettes:
\begin{itemize}
\item chemin 1: \texttt{\#A \#D}
\item chemin 2: \texttt{\#A}
\end{itemize}
La première instruction n'ayant pas l'étiquette \texttt{\#D}, et ne peut prendre que le chemin 2.
La deuxième instruction par contre pourrait éventuellement prendre les 2 chemins, on prend alors le chemin 1, car c'est celui qui a le plus d'étiquettes.
Si 2 chemins sont possibles pour une instruction, et qu'ils ont le même nombre d'étiquettes, alors il y a une ambiguïté et une erreur est générée.

\subsection{Partie implémentation}
\label{sec:timingDo}
La partie implémentation de la vue \texttt{timing} se situe toujours dans un bloc \texttt{do..end do}, comme dans la vue \texttt{behavior} et les composants. Quelques instructions sont spécifiques à cette vue et sont décrites dans les sous-sections suivantes.

\subsubsection{Ajout de cycle}
Cette instruction permet d'ajouter des cycle, et donc de simuler le temps qui passe: L'utilisation est: \texttt{add <expression> cycle}:
\begin{lstlisting}
do
  add 1 cycle
end do
\end{lstlisting}

\subsubsection{Instruction conditionnelle}
L'instruction conditionelle dans la vue \texttt{timing} est quasiment la même que dans les composants/\texttt{behavior}, excepté que la partie \texttt{elseif} n'est pas disponible:
\texttt{if <timingExpression> then <timingImplementation> [else <timingImplementation> end if]}. Il est à noté qu'il est possible de mettre soit une expression, ou bien de tester si un saut a eu lieu, à travers le mot clé \texttt{jumpTaken}:

\begin{lstlisting}
timing jumpTiming
  do
    add 1 cycle
    if jumpTaken then
      add 1 cycle
    end if
  end do
end timing
\end{lstlisting}
Dans cet exemple, une instruction met 1 cycle, et une instruction dans lequel un saut est réalisé met 2 cycles. Le mot clé \texttt{jumpTaken} peut être appliqué à toutes les instructions. Une instruction réalise un saut, si après son exécution, le compteur programme ne pointe pas sur l'instruction suivante: \texttt{ret}, \texttt{call}, \texttt{jmp}, \texttt{bra}, \ldots

\subsubsection{Gestion des erreur}
Il est possible de mettre une instruction d'erreur, de la même manière que décrit en \ref{sec:instError}, pour indiquer les cas qui ne devraient pas se produire.

\begin{lstlisting}
warning "message d'erreur"
\end{lstlisting}

\subsection{Accès à un composant}
\label{sec:timingMethodAccess}
Une partie implémentation peut être reliée à l'accès à un composant. ainsi, pour chaque accès à une méthode d'un composant, le nombre de cycle peut être mis à jour. On utilise pour cela une structure de type:
\begin{verbatim}
if use <componentMethod> then <timingImplementation> end if 
\end{verbatim}
\begin{lstlisting}
timing test
  if use sram.pop then
      add 1 cycle
  end if
end timing
\end{lstlisting}

De la même manière lors d'un accès en lecture ou écriture sur un registre, la structure est:
\begin{verbatim}
if read/write <registerName> then <timingImplementation> end if
\end{verbatim}
\begin{lstlisting}
timing testRead
  if read X do
      add 1 cycle
  end if
end timing
\end{lstlisting}
Ainsi, sur cet exemple, 1 cycle sera ajouté lors de chaque lecture sur le registre X.

\emph{Attention, il y a actuellement des effets de bord. S'il y a une session de debug, et qu'on affiche les registres, des cycles seront ajoutés!!}


%\part{Description de la micro-architecture}

\end{document}
